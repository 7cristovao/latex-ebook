%%%%%%%%%%%%%%%%%%%%%%%%%%%%%%%%%%%%%%%%%
% eBook 
% LaTeX Template
% Version 1.0 (29/12/14)
%
% This template has been downloaded from:
% http://www.LaTeXTemplates.com
%
% Original author:
% Luis Cobo (luiscobogutierrez@gmail.com) with extensive modifications by:
% Vel (vel@latextemplates.com)
%
% License:
% CC BY-NC-SA 3.0 (http://creativecommons.org/licenses/by-nc-sa/3.0/)
%
%%%%%%%%%%%%%%%%%%%%%%%%%%%%%%%%%%%%%%%%%

%----------------------------------------------------------------------------------------
%	DOCUMENT CONFIGURATIONS AND INFORMATION
%----------------------------------------------------------------------------------------

\documentclass[oneside,11pt]{memoir} % Font size

\input{structure.tex} % Include the file that specifies the document structure and layout

\title{Computador de 4 bits} % Book title
\author{Flávio Pereira de Oliveira} % Author
\newcommand{\edition}{Primeira Edição} % Book edition
%----------------------------------------------------------------------------------------

\begin{document}

%----------------------------------------------------------------------------------------
%	TITLE PAGE
%----------------------------------------------------------------------------------------

\thispagestyle{empty} % Suppress page numbering
\ThisCenterWallPaper{1.12}{littlered.jpg} % Add the background image, the first argument is the scaling - adjust this as necessary so the image fits the entire page

\begin{tikzpicture}[remember picture,overlay]
\node [rectangle, rounded corners, fill=white, opacity=0.75, anchor=south west, minimum width=3cm, minimum height=6cm] (box) at (-0.5,-10) (box){}; % White rectangle - "minimum width/height" adjust the width and height of the box; "(-0.5,-10)" adjusts the position on the page
\node[anchor=west, color01, xshift=-1.5cm, yshift=-0.4cm, text width=2.9cm, font=\sffamily\scriptsize] at (box.north){\edition}; % "Text width" adjusts the wrapping width, "xshift/yshift" adjust the position relative to the white rectangle
\node[anchor=west, color01, xshift=-1.5cm, yshift=-2cm, text width=2.9cm, font=\sffamily\bfseries\scshape\Large] at (box.north){\thetitle}; % "Text width" adjusts the wrapping width, "xshift/yshift" adjust the position relative to the white rectangle
\node[anchor=west, color01, xshift=-1.5cm, yshift=-5cm, text width=2.9cm, font=\sffamily\bfseries] at (box.north){\theauthor}; % "Text width" adjusts the wrapping width, "xshift/yshift" adjust the position relative to the white rectangle
\end{tikzpicture}

\newpage % Make sure the following content is on a new page

%----------------------------------------------------------------------------------------
%	TABLE OF CONTENTS
%----------------------------------------------------------------------------------------

\tableofcontents % Prints the table of contents

%----------------------------------------------------------------------------------------
%	INTRODUCTION SECTION
%----------------------------------------------------------------------------------------

\chapter*{Introdução} % Introduction chapter suppressed from the table of contents

\begin{quote}
Inteligência é a habilidade de evitar fazer o trabalho, e mesmo assim conseguir ter o trabalho realizado..\\
--Linus Trovalds
\end{quote}

É um belo lugar para escrever uma introdução ou um prefácio\footnote{Você pode até usar uma nota de rodapé para parecer mais inteligente}.

%----------------------------------------------------------------------------------------
%	BOOK PART
%----------------------------------------------------------------------------------------

\part{Computador de 4 bits}

%----------------------------------------------------------------------------------------
%	CHAPTER ONE
%----------------------------------------------------------------------------------------

\chapter{Apresentação}

Bem-vindo a uma nova seção do  canal nesta seção vou explicar  como um computador funciona no máximo  Baixo nível possível no qual vou me basear  para explicar isso no último projeto  em que eu estava trabalhando, que era o  projeto de um computador de 4 bits  baseado em integrado simples é  diga não complexo integrado  comportar flip-flops lógicos etc.  então sem muito trabalho entre  citações levam isso pode ser reduzido para  transistores simples  este primeiro vídeo como uma introdução  É uma apresentação simples embora para  a vantagem com isso é que com este solo  vídeo eles não vão entender como funciona  detalhe, mas vou explicar o que são  as características deste computador  suas partes constituintes muito acima  sem entrar em muitos detalhes depois para  no final vou deixar uma série de  exemplos para ver do que é capaz  este computador tão limitado e bom  Dê crédito às páginas onde eu pego  Ideias  Vamos ver quais são os  características deste computador aqui  é um pequeno documento do word para não  esquecendo nada de bom tem arquitetura  hardware o que isso significa que o  memória de programa e memória  dados são separados este tipo de  arquitetura é muito normal em  microcontroladores, como picks e  posteriormente em outro tipo de arquitetura em  o homem bonjour que é o mais normal em  quase todos os computadores onde há  uma única memória onde está  compartilhar programas com dados  digamos que embora o programa possa ser  em um tipo de rom é carregado em  o carneiro e o e ele corre e ele  compartilhar com programas parte  memória de programa tem um  memória cada programa pode ser  máximo 256 bytes e pode e pode ter  até 32 programas  as instruções são 16 instruções  regular o que quero dizer com regular  que todas as instruções têm o  mesmo comprimento são instruções de 8 bits  existem algumas linguagens de máquina onde existem  instruções mais curtas em  Neste caso são todos iguais destes 16  instruções existem 3 e implementadas é  diga que há espaço para três  mais instruções do tipo reeves de risco  Isso significa que é um conjunto reduzido.  de instruções de computador que mais  um ciclo de clock por instrução o que  Isso significa que toda vez que o  clube executa instrução em alguns  computadores é muito mais  precisa de vários ciclos para executar  uma instrução sobre isso não simplesmente  um loop executa uma instrução  essas instruções são separadas em  quatro bits os superiores são o código  de funcionamento, ou seja, dizem o que  tem que fazer enquanto os quatro  encores inferiores são o operando, ou seja,  o número os dados com os quais  operar aqui vemos o que é o conjunto de  instruções  bem a instrução 0 não faz nada  é usado para fazer atrasos, por exemplo, não  leva em conta o operando e é um  operação lógica a operação nam entre  operando e acumulador para 2 para ver um  pouco o que é o acumulador muito  rápido depois que ele tem uma operação  aritmética que é apenas uma soma  o que mais é a ideia cardiel operando em  o acumulador para isso é passado para  coloque o que está no acumulador em  a saida tem duas portas ai veja o que  mesmo  um porto de entrada de valor de carga  no acumulador tem uma porta de  entrada que os dados passam pelo acumulador  Digamos que o acumulador é onde ele vai  tudo agora venha todos os dados agora  Vamos um pouco mais sobre do que se trata  bem, esses dados que você diz ignorar  lucrar r AND b coloca o valor de  acumulador a no registrador de endereço  Isso ocorre porque, se você se lembra, temos o  instruções têm 4 bits que são os  opcode e operando de 4 bits  então 14 bits se quisermos  chamar um endereço da memória  só podemos ligar para 16  direções porque digamos que 2 a 4 são  16 mas se eu fizer as instruções  dos computadores estão em um registro  da parte inferior e, em seguida, adicionados 4  bits do operando  Posso acessar até 256 bytes de 255  256 porque o 0 também é incluído por  que diz que o programa tem até 32  programas de 256 bits depois tem isso  era coloca o valor do comunicador em  um recorde  este é um registro de uso geral que  você também pode ler e gravar  vemos como secar a carga o valor do  acumulador de registros para este outro  operação esses dois que não implementam  e estes são os saltos os saltos são  chama diferentes partes do programa jp  e é um salto incondicional e se  Eu chamo esta operação  por exemplo eu chamo jp e um valor vai para  salta para a localização da memória rd en  diga o que eu coloquei no  registrar mais o que aconteceu com ele no  operando e então tem dois saltos  condicionais o que é pular sim carregar sim  a faixa é ativada ou pula se o zero  então veremos um pouco mais  o que é isso quando eu explico a luz  não tem ram este computador dirá  como um computador sem ram soa como um  loucura e sim pode ser mas  Bem, a ideia é algo muito simples e  para que os conceitos sejam compreendidos  melhor no futuro eu posso adicioná-lo  ram que limita muito as coisas  o que o computador pode fazer  tem é o acumulador para acumular o  dados e, em seguida, este registro como o  chamamos de registros de área de uso geral  que também serve como um carneiro entre  citações para acumular alguns dados de 4  bits  o que mais ele tem, pois já vimos duas portas  de saída em ayerbe um porto de entrada  que nós os chamamos  também um registro de usos gerais e  uma avalanche de duas operações uma  aritmética que é a soma que vemos  aqui no app e uma lógica que é  o cndh pode não parecer assim e o  É verdade que continua muito, muito pouco, mas como  Eu sempre explico a eles que isso é algo para  entender como funciona, embora eles vejam  nos exemplos que podem ser feitos  algumas coisas bem interessantes  com um por exemplo com apenas um  operação lógica pode fazer algo  Eu adicionei isso vai escolher a operação nam  porque isso é o que se chama de  portão universal  então um portão anam eu posso  faça um portão não um portão  e um portão org short e todos os  portões podem ser gerados a partir de  se comportar e isso quer dizer que eles podem ser  fazer todas as operações lógicas, mas  a partir de etapas sucessivas e com o  adicione o mesmo com adição simples  aritmética pode ser feita outros  operações sim  aplicativos adicionam várias vezes o  uma exponencial é varias  multiplicações uma linha do  ponto de vista da aritmética booleana  também pode ser com uma soma  fazendo-o chamar o complemento de alguém  complemento de dois que eu acho que não  tem que explicar, mas bem, acredite em mim, eu sei  você pode fazer o mesmo quando não tiver ram  muito não pode ser feito  aritmeticamente com este computador  dois saltos condicionais que um é  faixas de faixa quando eu adiciono por  exemplo é calculado em quatro bits  o que significa que o barramento de dados é  4 bits para que possa ser armazenado  números de dados, por exemplo, que vão de 0  a 15 em hexadecimal seria de 0 a f  quando eu faço uma soma e essa soma  excede o valor f  por exemplo sumô 88 de 16 ativa um  sinal chamado carry  carry está em espanhol e então  também se o acumulador estiver todo em  zero ativa um sinal chamado  zero isso é feito para fazer saltos  condicionais se for zero que algo aconteça  parte do programa se você vir o ativado  carne ou jumper outra parte do programa  Isso é muito importante no projeto  de computadores muito bem com isso  há mais ou menos uma captura de tela do que  que as características que eles têm  como são as peças um pouco mais  divertido ver como isso é montado  computador, como eu disse antes, será  muito alto muito simples depois em  Próximo vídeo eu explico  exatamente como funciona, vamos pensar sobre  essa parte de cima eu fiz pc pc de  contagem de programa e contador de programa  o que ele faz que diz o que  parte do programa tem que ir como  é formado isso um relógio isso é um  clube veio se por exemplo 1 555  funcionando como estável e tem estes  dois integrados aqui que são contadores  Binários síncronos de 4 bits, ou seja,  que cada um desses quando chegar a hora  clube conta de zero até efe, mas ao seu  Tempo  o inferior é este vemos que tem um  saída  o que é isso sou eu que alimenta o  entrada deste outro é dizer que este  pode contar de zero a ff 255 por  é que os programas são limitados  para 256 bytes de 0 a 255 e eu disse e tem  32 programas porque esta memória é um  memória disso  8 kilobytes, ou seja, por isso são 64  estes podem ser acessados nestes 32  programas alterando a posição de  Como esses distritos veem, este é o  boa memória continuamos explicando em  esta memória onde o  programas são 8 casas 8 kilobytes de  memória  e bem vou explicar um pouco mais  isso informa o endereço de memória e  Isso é o que sai digamos em cada  endereço de memória há um valor de 8  bytes quais são quais são os  instruções reunidas pelo  opcode e operando  essa parte da rom tem essa outra  integrado que é uma árvore de buffer de estado  O que isso significa que o que sai de  aqui se eu não tiver isso anotado  ele vai para o barramento de dados se eu o tiver ligado  high não passa nenhum dado porque se não  como há muitas coisas que se juntam  mesmo barramento de dados não pode ser ativado  ao mesmo tempo, porque estariam à altura  curto-circuito digamos que se eu tenho um estado  alto se outro lhe der um estado baixo seria  um curto circuito  muito bom e como vemos  bom de passos já explicamos este é o  barramento de dados este azul o clube também  tem é composto e estes nós  comportar-se moldar um comportamento e aqui  tem um sinal de controle que diz  relógio  o que ele faz é enviar-lhe um clube  o que vai para o acumulador  Já que estamos falando do acumulador, vamos passar para  ver a luz os computadores que eu estava  vendo semelhante ao que você os projetou  passa em seu funcionamento em uma luz que  no 74 181 é bastante integrado  complexo e difícil de entrar  investimento muito mais divertido mais  didático faça o seu próprio alou para isso  é tão limitado a duas operações de adição  Inglaterra como esta forma desta luz e  Eu digo o que significa a luz, algo está vindo  de unidade  toda métrica e lógica para dizer que é o  parte do microprocessador que faz o  as operações aritméticas adicionam isso  a divisão neste caso apenas adiciona e  as comparações lógicas  Esse integrado que a gente vê aqui é um completo  sob isso o que ele faz é adicionar o que  entra aqui com o que entra aqui  se exceder esse valor, este é ativado  saída faz 4 que é o carry para isso  a gente vê que vem aqui  a gente vê que esse outro vem aqui  integrado aqui vamos ver e depois  finalmente o sinal de controle é ativado  que ele chama de carry  aqui temos quantos quatro portões  eles andam o que eles fazem na parte de  nando  então temos esse outro integrado que  É um seletor de dados que é o que eu  diz os dados que me interessam que  sair daqui são os da soma ou o que  Eles doam e eu trato disso com este sinal  controle que chama bar vê isso  Eu ligo para ela andy mais tarde, isso é para dizer sim  somar ou se eles andam  que mais como vemos a saída vai para isso  outro integrado que é integrado é o  que é chamado de tipo flip flow de octal  estado da árvore  Como eu disse, não vou explicar em  detalhe tudo vou fazer uma captura de tela  para quem sabe o melhor  entende muito do que eu sou  dizendo se eles não entendem eu não sei  se preocupe o próximo vídeo eu vou  explique melhor é octal  isso tem até 8 bits vou usar  apenas 4 é por isso que eles estão ativados  vai nesse caso são mais de 486 esses dois  você está de castigo e é isso que  faz é o que sai disso  deixe em mente e a lógica entre nisso  que isso em si é o que eu chamo  acumulador o acumulador o que ele faz  igualar os resultados e aqueles  os resultados voltam à luz aqui  neste caso na minha arquitetura eles fazem  através de um buffer que é isso que  vir aqui para dizer sim, dependendo do  sinal de controle que eu tenho quando diz  corrente, ou seja, do acumulador para o  luz passa dados para luz ou não passa  em muitas arquiteturas de computador  eles passam diretamente eu tive que adicionar  isso porque é muito simples e eu  complicou a questão de difícil, mas mais  nós temos esse retângulo aqui que  é o detector zero que faz  isto é, se é tudo zero ativo isso  sinal de controle finalmente chegando  aqui que no sinal zero para o  saltos condicionais vamos respeitar  carrega o zero  muito bem vamos seguir em frente agora vamos voltar para o  memória, memória, como eu disse, entre  um endereço de memória e gera um dado  que guardei na memória é o que  como eu chamo o programa  desses dados como eu te disse  as 4 últimas batidas são as  dados enquanto os 4 bits superiores  são as instruções  dita instrução deve ser  codificado por computador o que  faz essa parte das alças da CPU  é o que se chama de unidade de  controle, neste caso, isso faz parte do  unidade de controle em forças reais  ao controle que está todo misturado e  vamos lá porque esses buffett também é  parte da unidade de controle, mas bom  não é muito bom onde colocá-los bom o que  isso é o que chama o microfone com o  microcódigo que  que o que ele faz é de  instruções que te mando venha tomar  apenas os quatro principais dados  passar para esse integrado que é um  4 a 16 decodificador ou multiplexador  também é chamado assim o que torna isso  integrado é se eu prometer aqui um 1 para  exemplo 1000 aqui esse me excita  coloca isso abaixo de todos os outros estados  no alto enquanto eu não coloco tudo  em um isso me excita na saída 15  digamos que ele está aprendendo de 1 a 1  dependendo dos bits que eu envio para o  Entrada  cada um desses estados de 0 a 15  São precisamente as 16 instruções que  Eu estava nomeando ele se você vier aqui em  está nesta tabela terminar o  instruções nem boas a efe que  não está implementado se você o vir  faça o mesmo que você não entendeu  e se formos aqui não  Digamos que cada instrução desta é  implementado nesta coisa chamada micro  código como isso funciona depois de um  array que é basicamente o mesmo que  uma rom parecida com a que está aqui é  isso, mas me deixou muito mais caseiro  Achei muito interessante fazer assim  isso é bem gráfico, digamos que um  você quase pode ver como funciona, não é isso  estar dentro de um ônibus então  cada uma dessas barras é o  instrução as horizontais enquanto  que as verticais são os sinais de  controle, os referidos sinais de controle são  tudo definido como positivo por meio de um  puxe para cima e quando eu conecto um diodo e  Eu ativo este sinal, por exemplo, chego a 1  como vim aqui e eu tenho esse diodo que  significa que aqui neste sinal de  controle irá configurá-lo para 1 quando  ative essa instrução e aqui vemos  Temos 13 sinais de controle que não vou  para explicar o que cada um significa  isso é para nano água e explica o que  significa cada porque é muito longo  Por exemplo, aqui diz que vamos explicar  a on e, por exemplo, se isso estiver em  soma zero se estiver em 1 faz o  operação lógica, por exemplo este  jp e faz o salto incondicional é o  salto incondicional e assim por diante  esta última coisa que queremos aqui é  só tem com os saltos  condicionais o centro e condis é o  lógica que causa um salto para entrar  condicional  incondicional ou condicional vamos continuar  dependendo desses sinais de controle, existem  um sinal de chamada global de onde  este sinal local está chegando  entre aqui  esses contadores binários começam a  contar de zero a menos que eu coloque  um sinal aqui de 0 a 3 em cada  um e com esse sinal dar-lhe um pulso de  relógio colocar a perna para oad eu acho  em positivo não é negativo e de pulso  do relógio não importa tanto em detalhes que  significa que ele vai pular uma parte do  programa como o salto havia dito a eles  do programa é a parte inferior do  um registro de endereço que estes são  entradas aqui vão e conectadas ao  registro de endereço agora vamos ver  onde eles estão enquanto na parte alta da cidade  emite o operando que é esta parte de  aqui que está diretamente ligado ao  barramento de dados faz parte da memória  google parte dos dados se você estiver  ativado o buffer passa aqui e faz o  pule se eu tiver ativado  sinal baixo em louvor lá vão eles  entendendo um pouco do que se trata  este dos sinais de controle são  pulsos são zeros ou uns, dependendo  as condições e instruções que eu  dou-lhe  também muito rapidamente vamos ver  o que são registros  boa essa estrutura arquitetônica de  os computadores têm uma peculiaridade  muito raro que não vi em nenhum outro  computador, mas parecia muito mais  fácil de fazer é que este ônibus  que está em verde é o barramento de dados de  a do acumulador, ou seja, tem um  barramento de dados em azul e depois tem um  barramento de dados do acumulador normalmente  esses barramentos de dados estão juntos, mas eu  Eu fiz isso separadamente o que isso significa  que por exemplo para fazer uma saída  este é o porto de partida este é o  outro porto de partida ou o porto de  entrada tudo passa pelo acumulador  vamos ver o recorde desde  estávamos conversando há pouco sobre  registro de endereço  pega os dados do acumulador e vai para  esse integrado que está integrado no  que é chamado de um tipo de flop octal com  o dge me leve o que é chamado assim  então vemos o que é  basicamente o que ele faz é entrar no  dados aqui e salve-os deste  lado digamos enquanto o lazio que é  essa perna se destaca que eles veem como veem  tem um sinal de controle vindo do  micro código e aqui está rdl quando é  sinal está ativo permite que você passe os dados  se não estiver ativo, não importa o que eu  colocar deste lado que os dados fornecidos pelo  dados anteriores digamos os dados salvos  e como você pode ver isso é tal  mas meu computador de 4 bits, digamos  que isso é desperdiçado por aquilo  você é tão negativo e você é  as saídas não são usadas  o registro ao qual é o de uso geral  é exatamente igual ao registro  direção, mas tem a particularidade  que além de economizar dados posso  pegue esses dados e envie para  Como você pode ver, o acumulador tem dois sinais de  controlar a era o eixo que é o que  permite ativar o latch e o oe que é  aquele que permite ativar o integrado  Digamos que de certa forma enviar aqueles  dados para o barramento de dados e através do barramento  de dados para o acumulador vêm entre um data  do acumulador e depois do barramento  dados que também estão conectados ao  acumulado agora obviamente para a luz e para  volta da luz passa o acumulador é  carregar esses dados  este é o registro de uso geral que  mas aqui temos o mesmo integrado  esse outro também é amante de tudo isso  que são basicamente registros são  eles também são os mesmos integrados que  flip flop octal duplo integrado com  a borda  esta é a entrada como você vê aqui nós temos  os 420 ingressos que eu posso ir  Mudando  e tem um sinal de controle que é o  que permite que esses dados sejam passados para o barramento  dados em azul  então temos duas portas de saída  com estes organizados são exibidos bsd que  eles querem dizer que o que está andando eu vou  mostra em  em decimal em exa em hexadecimal neste  caso  sua porta de saída também é a mesma  tem sinal que é o katiba ou não  ativar digamos que se eu mano o que sai do  acumulador eu envio esta porta se eu  Eu ativei este sinal de controle  alto, neste caso, coloca em mim  lado aqui fora o mesmo para ele  porta para qual porta  Muito bem, acho que com isso está feito  uma captura de tela de todos os elementos  componentes deste computador  agora vamos ter algo muito mais divertido  o que é vamos ver do que é capaz  deste computador  este programa chama-se proteus atrás  simulações  Vou apertar play e vamos simular  agora eu carreguei com o programa que é  um contador que o que ele faz é contar  de 0 a ff digamos de 0 a 255 do  decimal  como a gente vê quando ele chega aqui em efe ele passa  conte adicione um número aqui vamos ver  efe lá ele foi para 2 o que ele faz é o  a conta é feita com somas simples  valor anterior soma-se e quando  ativa o carry pega o valor que está  no registro de propósito geral ao qual é  aquele que representa a porta a neste  caso de rosca e adiciona um também  sim podemos ver aqui  eles não vão entender isso bem, eles podem  faça uma captura de tela e tente  entenda que ak isso é um  programa o programa por exemplo vai de  aqui até aqui o que você pode ler isso  o que essas instruções fazem que isso é  o que é chamado de linguagem de máquina  que isso é o que eu sei por exemplo 34  Vamos ver se eu vou para a mesa aqui  o que estou fazendo é aumentar o valor  4 quer dizer que estou carregando no  acumulador a 4 e basicamente com isso fica  com o que funciona o computador  estas instruções que são regulares  todos eles têm um comprimento de 2 bytes como este  do que visualizá-lo em um editor hexadecimal é  muito simples nem precisa  passar por uma montadora e isso por  exemplo de contador de pontos bing se você vier aqui  vamos para aquele que está correndo nós consertamos  Já atingiu f 5 Vou pausar isso se eu  Eu clico duas vezes na rom é o que eu tenho  carregado no rondilla mos esse é o  programa que está funcionando contador  ponto de feijão  esse exemplo é bem simples vamos ver  alguns exemplos mais complicados já  vamos passar eu tenho eles abertos aqui isso em  Vamos ver este exemplo é como  controles através da porta para um motor  passo a passo  e com porta de entrada entrada  controlar a direção aí a gente vê que é  girando no sentido horário se eu mudar  a posição deste interruptor começa a  gire no sentido anti-horário e vice-versa  sentido horário anti-horário  Isso também se reflete em uma  programa que é isso mas aqui vemos  toda a linguagem de máquina que é o que  faz funcionar  vamos ver outro exemplo de lcd isso  computador poderia ter sido feito com um  único porto de partida, em princípio, tem  disse com uma única porta de saída, mas  Eu queria mostrar um exemplo onde eu dirigi  um lcd desses 16 por 216 caracteres  de duas linhas e para isso têm  uma maneira de trabalhar esse trabalho  com 4 bits, mas eles também precisam disso  placa  e o sinal daquele e como ele era mais  cabos resolvi colocar duas portas de  saída que nunca machuca e pode  controlar um display e eu vou te dar  jogue a simulação e vamos ver o  classe, vemos você escrever sinal de olá mundo  de admiração limpa a tela e  write entra em um loop infinito a  coisa muito simples  novamente isso se reflete em seu  programa que é esse que você vê aqui como  veja este olá mundo praticamente se  coma toda a memória a memória chega  até aqui de um programa sobre o tema da  que tem apenas dois contadores  binários de 4 bits, você pode colocar um  mais e depois de acessar praticamente  quase toda a memória, mas metade do  memória, mas para um exemplo  de como o computador funciona eu não queria  complique o com quantidade mais integrada  ok agora vamos ver outro exemplo  vamos para este teclado neste exemplo  o que estamos fazendo é bom eu vou  mostre como funciona diretamente  é um teclado matricial que possui três  bilhetes que vão para todos e quatro  sal e quatro saídas que ainda estão  entradas que é o que ele faz é um  varredura das entradas eu vou ver  algo clique em um e aqui na porta  me dá o valor que eu cliquei  enquanto a porta b eu uso para fazer  a varredura por exemplo eu clico 80  em 6 em 3  outros exemplos obviamente têm seus  programa em hexadecimal aqui você pode  veja se você quer uma captura de tela  como é formado  e vamos ver o último exemplo  o que é isso chamado de áudio que isso  exemplo é uma espécie de homenagem ao  primeiro computador científico  América Latina que era um computador  inglês a ferrante e mercury 2 que  Eu estava na cidade universitária do  Universidade de Buenos Aires nós somos  falando do ano mil 961 aquele computador  eles foram chamados clementine porque eles vieram  com um programa de exemplo  tocou a música da minha querida  tempo clemente minha querida clementine bom  Eu queria fazer com o meu computador um  homenagem vamos dizer e tocar aquela musica  neste caso como um dos exemplos  aulas com breu ou com ar de vinho  A música é gerada diretamente pelo  mesmo microcontrolador eu é assim  limitado que eu não posso fazer isso, mas eu faço  O que posso fazer é pegar um porto de  saída neste caso porta b passá-lo  por um multiplexador d um decodificador  idêntico ao que vimos aqui no ônibus  codificar e ativar essas notas e que elas são  carregado que o que eles fazem é dirigir um  555 no modo é estável e com esta rede  de resistência posso afinar cada nota  usando um pouco de matemática e  tocar música  esta simulação ao contrário do resto  não consigo fazer as simulações  em tempo real é o que eu faço aqui é  tem uma saída uma sonda de  tensão e isso ainda está ligado a isso que  chama a análise de áudio e faz uma  simulação e se eu reproduzir isso  simulação vai ouvir  tu  [Música]  o som é bastante áspero, mas eu sei  passa a reconhecer a melodia termina em um  excel mais tarde eu vou explicar se você  interessado em explicar como cada um funciona  um desses programas de amostra aqui  vemos a pontuação que vai por exemplo  isso é em Dó maior, mas vai do leste  sol a este sol é por isso que se você vier aqui veremos  que eu tenho oito notas variando de um  Sol  este é o sol mais sério para o sol mais  passagem afiada sem qualquer afiada  digamos que é uma grande escala de dedo, mas  que é organizado assim precisamente porque  é feito especialmente para reproduzir  esta melodia e bem e também temos  um céu que é silêncio se vemos em  O circuito  áudio aqui temos por exemplo  o 0 que é o silêncio depois de termos  todas essas notas eu ainda poderia colocar  mais algumas notas, como você pode ver, mas  bem, não era necessário para esta melodia e  foi um exemplo simples  ou tudo o que tem a ver com o  apresentação do computador o que eu  Eu preciso dar crédito a onde eu peguei o  ideias uma é esta página que é  bastante baseado neste computador e  é que é mais simples do que aquele que  Estou apresentando também não tem ram  só tem um acumulador não tem  um registrador de propósito geral então tem  um conjunto de instruções que são 8  3 bits e cinco instruções para o  operando e acessa apenas 32  instruções para seguir dois à quinta  não tem aquele cadastro de endereços que  Eu adicionei que torna possível acessar  até 256 instruções que é digamos  não é muito, mas é muito mais do que 32 que  além disso, esta luz do computador é baseada em  74 181 e parecia muito mais  interessante fazê-lo com luz própria  embora seja muito mais simples e mais  complicado e então o microcódigo  armazenado em uma memória rom igual ao  do programa ou um pouco menor  Parece-me um desperdício que por um  lado é bom porque eu em algo muito  compacto eu posso  Eu posso fazer do computador uma placa muito  menina parecia muito mais para mim  interessante esta outra solução para fazer  o microcódigo em um array que ele mesmo  é uma memória rom mas parecia uma  idéia melhor e também isso não se encaixa em um  rom porque eu preciso fazer sucessos para  exemplo para meu computador 13 sinais  de controle, então você precisaria de 2  memória rom apenas este computador  que eu estou mostrando a você que eles são chamados  com a chamada  mp4  apenas usado integrado ou dois ramos  para executar o microcódigo  em outro que me baseei menos que neste é  esse outro eles chamam e blair 4 cp univ  os possíveis são 4 bits exatamente  isso é mais poderoso do que o que estou lendo  apresentando-me o mesmo que o anterior  use como 'u'  74 181 afirma que eles veem aqui também  usa-os para microcódigo e tem um  contador binário mais para que eles possam  tornar os programas muito mais longos do que o  o meu está bem documentado tem  aqui bem tem aqui você pode ver através  arquitetura de chamada de exemplo venha aqui  Ele tem o acumulador que lhe contou tudo  entrar em algo que eu sou apenas  devido em um próximo vídeo farei  um esquema de arquitetura para  entender melhor e não ver diretamente  para o circuito depois é muito bom  documentado tem que baixar tudo é  também está muito bem documentado e  você vê que tem folhas de papel photo  que uma foto foi tirada e nada mais se uma  comece a ler em detalhes com isso  suficiente e farto para entender o  operação desta máquina 


%----------------------------------------------------------------------------------------
%	CHAPTER TWO
%----------------------------------------------------------------------------------------

\chapter{Descrição da arquitetura}

Bem-vindo ao segundo capítulo deste  nova série que estou chamando  operação de um computador 4  bits neste segundo capítulo vamos  ver uma descrição da arquitetura  no capítulo anterior apenas  ele disse que estava devendo isso  o esquema de arquitetura aqui  nós obviamente vemos a página como sempre  vou deixar o final do video para deixar o  link para a página e na página eles vão  poder descarregar ou consultar esta imagem  basicamente com esse esquema que é um  descrição de uma arquitetura  computadores na ideia de que  conceitualmente é melhor compreendido  funcionando do que vendo diretamente  para o circuito  Vamos começar a analisar o esquema de  de cima para baixo  Este primeiro quadrado que estou marcando  com o mouse é o contador do programa  o que o contador de programa faz é  diga à memória do programa para  estude outro retângulo chamado rom a  qual endereço de memória ele tem que  acessar  o que essas setas indicam são os  ônibus e o número dentro indica  o número de bits aqui, vemos que isso  inicializa este barramento tem 13 bits que  basicamente se formos no circuito que  vimos no capítulo anterior são estes  direções que vão do aço ao  as 12  como essas direções são formadas os 4  bits inferiores vêm do registrador  rd que eu nomeei um pouco no capítulo  anterior próximo 4 batidas vêm  do barramento de dados que é este barramento  grande vertical aqui e os 5 bits  estes últimos vêm de alguns mergulhos da praia que é  o que eu te disse no capítulo  acima que pode ser armazenado até 32  programas que vêm desses 5 bits é  diga 2 à quinta de até 256  Mordidas que são exatamente essas 44 dão 8  bits 2 a 8 são 256  esta é outra seta menor junto  com um texto que é visto em todo o  esquema por exemplo aqui está o que  Eles são chamados de sinais de controle.  normalmente em um esquema que  descreve a arquitetura de um  controles de sinais de computador não  estão colocados, mas pareceu-me que o  interessante colocá-los porque  mais tarde, quando eu explicar tudo mais em  detalhe me parece que são  aqui muito bem disse isso vamos continuar com o  próxima parte do esquema  esta próxima parte que eu já falei sobre um  pouco é a rom ou memoria do programa  o que ele faz é inserir 13 bits informa  qual endereço você tem que acessar e que  endereço de saída de um byte  esse byte é dividido em dois  os quatro superiores que vão ao micro  código agora vamos ver o que é o micro  código e os quatro inferiores que vão  para o barramento de dados, obviamente, à luz de  os dados não podem ser acessados diretamente é feito através de um buffer que é  este losango está aqui com seu sinal de  controle correspondente e quase  habilitar ou desabilitar  o microcódigo é o dispositivo que está  responsável por decodificar esses quatro bits  ou seja, você pode ter até 16 anos  combinações e gerar os sinais de  controles que permitem gerenciar todos os  computador digamos misturado entre  tudo isso é o que se chama unidade  controle em microprocessadores  vamos ver no circuito o microcódigo  foi isso se você não se lembra desse  matriz de diodos que exatamente o que  faz a decodificação com um dos  multiplexer e bem, aqui estão alguns  portas lógicas usadas para  faça os saltos digamos os sinais  O que temos aqui são mais esses 13 sinais  o sinal do lugar que é o que indica  para o contador de programa de um salto  direção  bem, vamos ver com quem mais continuamos como  ele disse vamos de cima para baixo  Bem, já falamos sobre a rodada do micro  código contador de programa  Agora vamos passar para alul alóu no  unidade métrica lógica de Allen no vídeo  anterior eu disse a você que esta unidade  para a métrica lógica há apenas dois anos  operações que é somar e faz um  operação lógica que está dando um  comparação com a luz que entra no ônibus  dados  precisamente os dados que entram da parte  baixar as instruções e, em seguida, o  outra entrada faz isso através deste  que é chamado de acumulador normalmente  o acumulador é conectado diretamente a  luz neste caso por uma questão de  hardware eu não poderia fazer assim e eu tenho um  buffer intermediário, digamos, para que  passar os dados do acumulador para a entrada  da luz tem que ser permitido  este buffer  bem, basicamente, 4 bits entram aqui  4 dias aqui e sai a operação que  Eu disse a ele para subir e por sua vez  para isso precisamos de um  ciclo de clock que o ciclo de clock sai  daqui do contador de programa venha aqui  temos uma flecha que sai que é o  clube  acne e outra seta que diz clock a que  venha aqui enquanto aqui temos um  outro sinal encontrado que está bloqueado  que é habilitado o que sai do  unidade de controle você vê aqui no  circuito que é refletido aqui relógio  aqui ele passa por uma porta lógica é um  e isso emite esse sinal que chama clock  que é o que finalmente entra aqui para o  acumulador eu digo isso rapidamente para  dê uma captura de tela e pronto  mas depois eu explico  tudo em detalhes  vamos seguir esses retângulos que você vê aqui  acumulador out a out b área e a r são  todos os registros  Eles têm usos diferentes, mas em si mesmos são  todos os registros que este computador tem  uma característica bastante estranha  é que para um dado passar por um registro  Eles não conseguem passar por este ônibus que eles veem  aqui estou marcando com o mouse que  É um ônibus que eu chamo de madeira o  acumulador no circuito é esse que  está em verde normalmente os calculados  o que eles fazem é isso eles o mandam para o ônibus  dados e o barramento de dados vai para cada  registro foi mais fácil para mim  faça assim um pouco do que com o clube  Bom, um meio complicado de explicar  vamos ver se vemos depois  ok, então digamos que isso é o que  faz é que o acumulador sai do  dados e, por exemplo, se eu quiser tirar  algo no porto para  passar do acumulador para a porta para  usando esses sinais de controle para  porto vê a área como vemos no  a porta a e a porta ve são apenas  saídas, então talvez eles tenham dois  sinais de controle para ativar o  saída enquanto a área que é a  registrar que chamei de uso geral  É entrada e saída e como vemos  Possui dois sinais de controle, um para o  saída e outro para a entrada e por isso  Larrea também está conectado ao ônibus  dados  se eu salvar um dado na área  que vem do acumulador e então eu  quando eu quiser eu pego e passo  para o barramento de dados o registrador de entrada  É apenas entrada para a corrente  tem apenas um sinal de entrada e  que vai diretamente para o barramento de dados  vai o barramento de dados depois da central para  algo para o contador do programa onde  tenho que ir  que mais finalmente temos o rd que  também passando pelo acumulador de carga  diretamente na parte inferior do  endereço no contador de programa  basicamente isso é tudo, digamos com  o que eu vou dizer está feito  descrição geral do que é  arquitetura deste computador antes  para dizer adeus, eu queria comentar que há um  erro nesse desenho e a verdade me dá  um pouco de preguiça de gravar o video todo  para trás esta seta que está aqui que  entrar no cali é uma saída se vemos aqui  vem é uma saída do acumulador sai  dados e voltar à luz para  do buffer ou para gravar  correspondente bem aí você vê a  diferença. Este é o esquema


%----------------------------------------------------------------------------------------
%	CHAPTER THREE
%----------------------------------------------------------------------------------------

\chapter{Contratador de programa}

bem-vindo ao capítulo 3 sobre o  operação de um computador 4  dias neste capítulo das coisas já  eles estão impondo um pouco mais  interessante porque eu vou explicar que é como cada uma das partes funciona  que constituem este computador hoje  vamos começar com este quadrado aqui  o que é que eu fiz pc o que é o contador  programa contador de programas  basicamente o programa sing é o que  informa ao computador qual endereço  memória tem que acessar vamos ver  O circuito  esta parte está aqui em cima, vemos isso  aqui na memória que na próxima  capítulo explicar mais em profundidade  como funciona, mas basicamente como  a gente vê aqui tem 13 bits esses 13 bits 8  sai do contador de programa e o  últimos 5 você sai de um interruptor isso  digamos que é algo mecânico  Agora vamos ver exatamente como  Esses CIs funcionam?  essas portas lógicas que você carrega  entender que vamos passar por outro  circuito um pouco mais simples do que  é isso, embora seja basicamente o mesmo  temos esses dois circuitos integrados  quais são os 74 hc 161  Eu tenho a folha de dados aberta e  vamos ver vice quem é contador  binário apresentável síncrono de 4 bits  com um reset assíncrono muito bem vamos  vamos ver o que exatamente isso significa  Para entender isso, vamos voltar para  circuito simplificado  bem, vamos jogar diretamente para  circuito para ver a simulação  tem esse pedaço que eu tenho aqui é o clube  quando há uma borda de subida, ou seja,  de 0 passa 1 veja aqui aqui contado em  binário 1 se eu clicar novamente  234 e assim por diante poderia fazê-lo  conte e isso é obviamente  hexadecimal mostra a amostra para  exibição  uma vez que eles chegam, digamos no  o próximo bit já seria tudo isso em  alto valor vamos ver que o pino rc ou  fica alto também aí chega  efe e o rc ou o alto é colocado em algo  então esses contadores binários que  Eles são 4 bits apenas para poder produzir  são 4 bits podem ser colocados em cascata  como você conecta em cascata isso  rs ou com esse outro pin aqui chamado  e em que vemos que é alto agora  precisamente porque estão interligados e  quando eu der um pulso de relógio novamente  vamos ver o que acontece lá ele deu-lhe um pulso  relogio o rs o zerou esse  voltou a zero e foi para um  vamos fazer alguns cliques  e a mesma coisa acontecerá novamente a b c d  e mais um clique é definido como 1 e quando  próximo clique  isso vai para o  muito bem  com esta primeira parte entende-se  muito bem como funciona um contrato  comum e como ele está conectado em  cachoeira  por outro lado, eles têm um pino de reset que  se conectou ao positivo se eu fizer  clique  redefine imediatamente não precisa  espere um ciclo de clock ou qualquer coisa para isso  estilo como vimos aqui dizia que o reset  foi um síncrono, pois há outra palavra  mais do que diz que está apresentável vamos  veja o que isso significa  o que é muito importante para o  computador  os computadores precisam fazer saltos  do programa  programas são geralmente sequenciais em  em grande parte vão do endereço 1 ao endereço  2 a 3 a 4 e aí você tem que fazer  saltos saltos para suas rotinas ou saltos para  outras partes do código como fazer  aqueles saltos apenas com a parte  apresentável  muito bem, vamos deixar isso em jogo e  vamos continuar assistindo  Eu vou fazer alguns cliques isso é  em 6 digamos que agora eu vou fazer isso é um  pule para outro lugar enquanto faço isso com  Esses beats aqui, por exemplo, eu vou  fazer efe jump, por exemplo, como o  pular para efe colocou tudo isso em f  vou ver agora vou dar uma  ciclo do relógio e não vai para efe  mas vai acontecer ou vamos fazer  pular para a vida  melhor que 2 por exemplo você está em 12  digamos um loop que não vá para 7  por que isso acontece porque para ele dar o  salto eu preciso deste alfinete este alfinete  em vez disso, é chamado de carregar isso em um  valor baixo agora sim sim defini-lo como zero  ainda em 7 me segue mas quando dou um  pulso do relógio observe o que acontece pule  para 2  efe é dizer o que tenho aqui se eu continuar  dando pulsos de relógio  nada acontece porque porque este continua  ativado em baixo, então toda vez  dado pulso de relógio eu carrego o que  Eu tenho este lado, mas se este o colocasse  em voz alta continuar a contar por uns para  de dois f  muito bom muito mais para explicar não tem  sobre o balcão o que vamos  explicar isso que está aqui que é um  portão e  que está conectado  mas para explicar isso você já tem que  Lugar, colocar  um pouco mais aqui  se vemos no circuito é isso que é  aqui  que existem dois portões nand isso  é equivalente a dizer que é uma porta  andy  vamos voltar para o outro circuito  para que serve isso porque às vezes eu  clube isso eu preciso que eu saia sempre  contando aqui e às vezes eu preciso do  clube vá para o acumulador é por isso que isso  ping diz aqui relógio a como você fez isso  definindo este bit alto se eu definir  este vídeo em alto quando eu cronometrá-lo  aqui olha agora como vai piscar  isso acontece toda vez que coloco  alto também passa alto olha lá vem  como você entra  em vermelho  basicamente isso seria o suficiente  bem explicado na minha opinião não espero  ter entendido como o  contador de programa  podemos adicionar algo mais ao  explicação eu acho que sim como ele tinha  disse que este computador é baseado em tudo  em circuitos integrados simples que  Eu disse em algum momento que isso mesmo  poderia fazer com transistores com um monte de  paciência não então não sei como  esses circuitos são constituídos  integrado  Se voltarmos à folha de dados do  circuito integrado e continuamos descendo  você pode baixá-lo para a folha de dados e  leia tudo exatamente como funciona i  Eu expliquei algo muito rapidamente, já temos um  esquema  mas o esquema que mais me interessa  bem, vamos descer muito mais é isso  como você pode ver, isso nada mais é do que comportas  lógico como vemos aqui e flip flops um  flip flop é o que se chama de  batida estável não vou explicar agora  muito que é um flip blog, mas eles são  basicamente um elemento habernos abel  Veja aqui o que eu fiz  é um vibrador múltiplo capaz de  ficar em um dos estados  possível por tempo indeterminado  ausência de perturbações digamos que é um  dispositivo eletrônico por assim dizer  uma forma que permite manter um banco de dados  digamos que é como uma memória de um bit  existem muitos tipos de bi estável como  logan aqui eles dizem rs de jk t de tudo isso  Eu diria a eles para lerem sobre o que é aqui  na wikipedia o ambiente é bom  explicado e se eles vão para a wikipedia em  Inglês é muito melhor explicado  certamente na página de vídeo do youtube e  outros explicam exatamente como  trabalhar algo que eu queria ir é isso  esses bi estáveis ou esses flip flops vão  Ele os conhece até em espanhol.  geralmente chamados de floco de movimento também são  formado por compostos lógicos como em  aqui é formado por  portões não por exemplo  este é exatamente o mesmo  formado pelos portões dando  por que eu te disse isso  para que se entenda um pouco melhor como  Digamos que você se comporte logicamente  se você olhar na internet eles são formados  por transistores não é algo muito  É difícil para eles, que é quando a gente coloca  vários estão adicionando o campo lógico de  o número de transistores e estes  retângulos que representam o flip  fracasso  isso também pode ser feito com isso  integrou o 74 hs 74 que é um dual de  widget de fluxo na reinicialização, digamos isso  está integrado são esses quadradinhos  Individual  o que eu fiz  neste circuito também abaixo  acabei de fazer um contrato normal  bem mais simples que este  aqui é basicamente essa parte que você vê  aqui está um contador binário  Ascendente de 4 bits para mostrar que  pode ser simples pode ser feito  simplesmente como você vê  Vou pausar o jogo novamente e  vemos que a contagem 12 agora tem um clone  de cada um segundo a perna de conecta  ao negado o q negado o q na saída  e, por sua vez, o kun negado se conecta ao  clube de outros e feeds serão relacionados  aquele que negou com os dados e este é o  redefinir perna que também neste caso  é um reset  digamos síncrono eu clico e  redefine para zero  até eu colocá-lo de volta para 1  ainda em 0  com isso sim seria toda a explicação  O que ele estava planejando fazer sobre isso?  sobre o contador de programa ou bom  mais uma coisa  Este integrado, como eu disse, é um dual  de bits de tanque de fluxo livre definidos sim  também descemos na folha de dados  veremos que eles não são mais do que se comportam  lógico também  Digamos que tudo isso pode ser feito  média para a expressão mínima com  bastante consciencioso e bom e entrando em  e tentando entender como cada um funciona  um desses se comporta lógico

%----------------------------------------------------------------------------------------
%	CHAPTER FOUR
%----------------------------------------------------------------------------------------

\chapter{A Memória de Programas (ROM)}
bem-vindo ao capítulo número quatro  sobre o funcionamento de um  computador de 4 bits neste capítulo  vamos ver como funciona a rom  ou seja, a memória do programa esta  integrado que vemos aqui antes de dizer com  isso faz uma ressalva  Como eu disse no capítulo  1 quando fiz a apresentação desta  computador tem sido uma arquitetura que  é chamado de tipo de hardware onde o  a memória do programa é separada do  memória de dados em mais do que mesmo  tem ram a memória de dados são  registros simples neste computador são  uma saída importante é porque se  estamos tentando entender como  um computador funciona é algo muito  importante ter em mente uma vez que o  a maioria dos computadores, como  você está assistindo este vídeo  computadores tablets celulares qualquer coisa  se em uma arquitetura que chamamos  newman onde a memória de dados e de  programas são compartilhados  da mesma forma que fiz no  capítulo anterior para entender como  memória de programa funciona vamos  vá ver um circuito simplificado  como vemos a memória do programa em  este caso diz 27 64 este é o circuito  integrado que estamos usando sim  vemos na folha de dados que diz que 64 k  v eprom atp rom isso o que significa 64  aqui vamos ver de onde vem  Como podemos ver, ele tem 13 bits se eu abrir o  calculadora  Vamos colocar o modo científico e se  fazemos 2 elevado a 13  me dá exatamente os 8k que fala  esses 8 k são 8 k byte igual a 64  bits digamos para cada endereço  representa um byte 8 bits  v porque este tipo de memória é  pode apagar por luz ultravioleta ou este  que é você, mas isso significa que é  você pode programar apenas uma vez que é  que uma vez programado não sabe  pode apagar  Estou usando esta memória porque é  memória na qual você pode simular o  proteu  na área você pode simular um monte de  memória, mas dentro dos que eu tinha  destinado ao design case-in-case  para construir este computador, se houver  dia eu queria construí-lo com certeza  Eu usaria essa outra memória que é muito  semelhante, mas é um baile de formatura, ou seja, o  quadrado primeiro, quero dizer que você pode  apagar eletricamente  vamos ver como essa memória funciona  vamos voltar para a folha de dados  memória que estamos fazendo  simulação se virmos o pino que temos  13 pinos de endereço 8 pinos de saída  dados um pino de habilitação de chip  um pino de habilitação de saída a  pino de programação para alimentação e  o resto vamos ver o circuito  simplificado  é a memória do programa de modo que  funciona obviamente tyc baixou um  programa na simulação no proteus  como fazer duplo clique  e se formos aqui tem um programa de lhamas  programa dot que é um arquivo binário  basicamente  vemos este programa aqui, temos que  na posição de memória 0 temos 0  Digamos que eu tenha 01 na posição 0 no  posição 1 tenho 02 na posição 3  Eu tenho 04 08 10 2 14 e 18 etc. e em  a posição  efe efe eu tenho aa e depois o resto  eu tenho tudo sendo f  Digamos que aqui está o que estou acessando  são para a memória destes que podem  acesso com esses 8 bits é isso que  Eu acesso com o contador de programa  Eu uso essa grande memória porque é  aquele que simula proteus e porque também  comercialmente fácil de obter  há memórias menores sim, mas não  Eu sei se eles podem ser encontrados - aqui na Argentina  Agora vou apertar o play pra ver o que é  o que acontece aqui estou acessando o  memória como vemos efe efe que é  onde tenho no editor hexadecimal a 6  0 1 0 1 0 1 etc. e se vemos no  prote apenas mostre aqueles 0 1 0 1 0  1 0 1 0 1 agora vamos por exemplo para  local de memória 0  e vemos cada um temos 11 chegamos a  eletro decimal extra temos um 1  se eu for para a posição de memória 2  tenho  para a posição de memória 1, temos 2 para  a posição de memória, por exemplo, 4 para  posição de memória 3 temos aqui um  4 etc. temos esses dois chips esses  dois pinos que são a habilitação do  chip e habilitação das saídas  que ambos têm que estar para baixo para  o nível baixo, quero dizer, para que  mostrar a saída, então temos o  poder e chip e  programando a programação nós  não usamos porque estamos simulando  mas ao construir isso  fisicamente primeiro vamos ter que  gravar com um programador esta memoria  basicamente não há muito mais do que isso  explique como funciona a memoria  tem uma direção e para isso  endereço tem uma saída de 8 bits  esta memória o que eu gostaria  explique um pouco o que é isso  editor hexadecimal que para alguns  pode ser extremamente óbvio para  outros talvez não, então eu vou  explique que este é um editor hexadecimal  por cabo que tira da internet  não é realmente chamado h  portátil xd  é um editor hexadecimal o que você  vamos fazer é ver o conteúdo  binário de um arquivo por exemplo eu  Eu fiz este arquivo no bloco de notas  que eu os chamei de hello dot txt e o  conteúdo ou a letra maiúscula se eu  Eu falo em um editor hexadecimal que posso ver  o conteúdo precisamente de cada um dos  esses dois bytes cada letra cada caractere  é um byte se eu falar por exemplo o  mapa de personagens  mapa de caracteres  Vamos ver isso para o h maiúsculo onde  Eu tenho que corresponde ao personagem  hexadecimal este 0 x o que significa que  está em hexadecimal aqui onde este eu  norman 48 e como vemos aqui no editor  hexadecimal aparece 48 eu sei ao  letra maiúscula corresponde a 41 vamos continuar  ao editor hexadecimal veremos que um  a letra maiúscula corresponde a 41  Isso permite que você trabalhe com o  arquivos exatamente com o que está lá  dentro podemos editar tudo  Espero que com esta explicação  rapidamente e facilmente foi compreendido como  memória de programa funciona

%----------------------------------------------------------------------------------------
%	CHAPTER FIVE
%----------------------------------------------------------------------------------------

\chapter{Unidade Aritmética Lógica (ALU)}
bem-vindo ao capítulo número 5 sobre  o funcionamento de um computador  4 bits  Neste capítulo vamos ver a luz deste  retângulo que você vê aqui que a luz a luz  na unidade aritmética e lógica que  vem uma saída de inglês é  apenas aritmética lógica e unir  como seu nome indica ao lou no  parte do processador que lida com  executar lógica e  aritmética quais são as operações  lógico por exemplo e nan ou curto não  etc. esse tipo de portas são  todas as operações lógicas e aritméticas  adição subtração multiplicação divisão  Muito bem, vamos continuar, eu sei quais nuances de  desta forma onde temos dois valores de  entradas neste caso existem b que dão um  resultado por exemplo pode ser ambos  por sua vez tem um sinal de controle cada  está marcado como efe que é aquele  indica, por exemplo, que tipo de operação  faça se faça uma soma entre esses dois  valoriza algum tipo de comparação  etcetera e também tem sinais de  saída que veremos mais adiante  isso é  precisamente se vamos ver o esquema de  minha arquitetura de computador  esta é a luz como vemos temos dois  entradas de quatro rebatidas e uma de  4 bits que é o resultado da  operação a ser feita e por sua vez tem  seu controle sinaliza tanto a entrada  como saída, as saídas passam  o acumulador depois depois vamos  para ver porque  vamos voltar ao círculo completo  Como eu disse no circuito este é o  a luz é uma saúde projetada por mim existe  para comerciais leves que vêm em um único  chip, por exemplo, o mais conhecido dos  74 181 é uma boa opção se você  eles querem fazer algum desenvolvimento, mas como  este é um computador  didático muito parecido muito mais  interessante mesmo que menos poderoso  desenhe o seu próprio  olá o que é isso que você vê aqui como o que  fiz em outros vídeos para entender como  funciona vamos ver um circuito  simplificado  Isso é algo que eu projetei, é tão simples que  tem o mínimo necessário para poder  ser é chamado de luz que tem um  única operação lógica que é nando e um  operação aritmética única que é adição  como você consegue esses 4 bits é uma entrada  esses outros 4 bits que são  representados para cada um em um  exibir na é a outra entrada e estes  quatro daqui na saída dizem o  ace operação feita entre esta entrada  e esta entrada vamos ver um pouco  como funciona  basicamente temos como eu disse que temos  duas operações adicionar como operação  aritmética e nan como operação lógica  da soma se encarrega de fazer isso  integrado que é o 74  hc 283 que é o que se chama full house  ader vamos ver a folha de dados  o que é um ader completo um ader completo é um  apenas um circuito lógico daria  seria uma boa aritmética, mas é baseada em  circuitos lógicos para funcionar que  faz a soma de dois valores como  veja aqui temos  um valor de entrada de 4 bits outro valor  a entrada de 4 bits e um valor de  A saída do acrônimo é a soma deste  valor mais este valor e então temos  esses outros 2 pinos esse é o trilho  dentro e o outro é a pista auto lane é  algo como carry seria o  tradução em espanhol ou transporte  exemplo mais simples porque então quando  isso somamos dois valores as vezes ficamos  estouro, digamos 4 bits em  4 bits o número máximo que podemos  representa é um 15  o que seria efe em hexadecimal então se  vamos ao circuito por exemplo aqui eu  Estou adicionando vamos adicionar dois valores  saber  Vamos adicionar 9 9 9 9 18 em decimal não  mas eu só tenho 4 bits  Com licença, vamos adicionar, lá estava  fazendo operação nand lógica agora  está somando  9 + 9 como eu te disse 18 seria 2 em  hexadecimal vamos ver o calculado  científico vamos colocar hexadecimal  adicionar 9 + 9  é igual a 12 em hexadecimal não em  decimais é obviamente 18  mas o computador não pode adicioná-los  valores porque transbordam ou em  na verdade sim para como você faz isso quando  esse valor excede este pino é ativado  o que a carrie diz aqui precisamente isso  depois desse sinal de controle de saída  em que usamos para fazer somas  maior do que é o que a luz permite  que são números muito pequenos, por exemplo  vamos remover esta parte e estamos  somar 1 + 9 seria 10 estaria em  hexadecimal e como vemos  sinal de controle de faixa saiu  mudar se eu colocar de volta o 98 por  exemplo lá me daria 11 em hexadecimal e  sinal de carga acende  Como eu disse, esta é uma casa cheia  veja que é um cheio tem muito não vou  para explicar vou passar um pouco  up, mas eu vou mostrar isso, então se  você quer entender melhor  exemplo na wikipedia que eles procuram ver desde  explica como funciona é mais até  Tem algumas animações é o que eu sei  chame meia água ou meia der e  então há o lenço que o full a  veja o que você tem é exatamente isso  sinais desses sinais de carvão e  realizar que são muito importantes porque  porque isso permite que ele se conecte em  cachoeira, ou seja, por exemplo, que estamos  transforma um computador de 8 bits em  eles não são realmente vendidos a baixo, acho que não  existe comercialmente o chip que  vem cheio pra ver 8 bits mas o por  o que por que não porque não é importante  diretamente se um conectando dois de  esses chips e conecte o carro de transporte  do primeiro ao carbono do segundo  eles se conectam como se fossem uma cachoeira e ali  temos um ader completo de 8  como eu sempre digo este computador  pode ser reduzido ao mínimo e  se virmos na folha de dados um completo a  Vejo  Não é nada mais do que se comportar logicamente, eu gosto disso  sempre enfatizar que ele fez como pode  volte e entenda exatamente  como tudo funciona  mas com isso já que de lá explicou para  a soma digamos a operação aritmética  Agora vamos ver a operação lógica  nand como fazemos essa operação  lógica cndh muito simples com quatro  portões andando vamos ver o que é o  operação nam aqui temos portão nand é  na tabela verdade do portão  e digamos que sempre que houver um  zero vezes um 1 digamos que é como um an  negado  há sempre um zero para um com o único  A maneira como a saída aparece como zero é  que ambas as entradas são 1  vamos ao circuito simplificado de volta  antes de entrar no circuito  simplificado porque escolhi este portão  porque o portão é o que é  chamado um tipo de portão universal  o que isso significa que eu com um  portão anand eu posso fazer qualquer  outro tipo de portão por exemplo  conectando-o desta forma eu tenho um  portão não está conectando o deste  outra maneira eu tenho uma caminhada de portão  conectando o desta forma eu posso ter  um portão por e assim por diante  pode fazer qualquer tipo de portão  começando apenas de se comportar para isso  computador não tem memoria ram então  o que é meio complicado de fazer  operações lógicas muito longas, mesmo se  você pode usar algum log o log  uso universal e outros podem ser feitos  tipos de operações lógicas iniciando  eles se comportam, digamos não  precisamos de todas essas operações para  realmente fazer uma operação lógica de  outro tipo, mas é feito em etapas  sucessivamente e salvando os resultados  bem o mesmo pela forma como cheira um pouco  à soma acontece o mesmo com a soma  exatamente o mesmo  digamos que é calculado apenas  pode acrescentar e não a verdade que não pode  fazer muito mais operações porque por  exemplo multiplicação são adição  sucessivos ou seja, da lição  mais pode ser feito  calmamente multiplicar a linha e  a subtração mara do signo não é uma soma  pelo contrário, na verdade eu posso  diga se é uma soma neste vídeo não não  Vou explicar a aritmética binária, mas  por exemplo, se você estiver interessado  pesquisa sobre o assunto para encontrar um  um pouco mais procure por exemplo o resto  você usa o que o plugin chama de  complemento de 2 ou 1, que é uma forma de  fazer subtrações de adições embora  Parece meio estranho, acredite, é assim mesmo.  divisão o mesmo  digamos que pode ser feito de  somas simples podem fazer muito  de operações aritméticas e de  o portão anand pode ser feito um  bando de operações lógicas  o que mais nesta página eu encontrei  vou te deixar o link depois  É interessante como posso te dizer que isso  pode ser reduzido a transistores aqui  vemos como eles se comportam  a partir de transistores, digamos que é  algo como em algo bastante simples  ok vamos ver a folha de dados a  pouco de está se comportando é o  integrado em 74 200  não tem muito o que mostrar são quatro  comportar-se exatamente como eu sou  precisando  que voltamos ao circuito  simplificado e temos esse outro  integrado aqui no 74 157 agora vamos  ver o que o torna bom, eu vou repassar  explique  Esse integrado que a gente vê aqui no 74 283  é aquele que o cheio de ver aquele que faz o  soma e é integrado é sempre  adicionando e, por sua vez, esses portões  andan sempre faz a operação lógica  nan e então o que temos é isso  integrou os 74 157 que desde  nós colocamos este pedaço  emite a soma ou o  operação lógica nós vemos lá quando  está em zero está adicionando enquanto  se colocá-lo em um não está fazendo nada  Como vemos isso, dissemos não  coloca um zero apenas se ambos forem  um ou sempre que houver um zero vezes um não  Digamos que aqui temos na saída de  isso temos 11 10 ou seja 7  Ok, vamos olhar para a folha de dados.  desta integrada a h a 74 hs  157 é o que chamamos de seletor  com duas entradas  temos um byte de nível médio 4 bits  um nível de entrada para outro nível e sai  um destes dois níveis dependendo  qual é a seleção que fazemos o que  esse pedaço que a gente vê aqui  como vemos é muito simples  bem, precisamos ver todos esses  portões que estão aqui você está se comportando  que estão aqui não passa de um circuito  lógico que o que ele faz é detectar  quando o resultado é agredido eu tenho  um zero isso é muito importante porque  esses dois sinais de controle carregam  o carry como o zero nos indica em  a  no código, faça saltos condicionais  diga para passar um ponto de código  um local de memória para outro quando  ser algumas dessas duas condições já  ser zero ou carregar colocamos zero novamente  e vemos que não está ativado porque o  resultado não é zero vamos fazer isso  o resultado é zero  colocamos tudo em um  e precisamente o resultado é 0 e é  ativa esta esta saída este sinal  controle de saída bom não há muito mais  o que explicar sobre a luz  se voltarmos ao circuito completo  Vamos ver que dentro da luz temos  também esses dois integrados não eles  Estou explicando neste vídeo porque  isso escapa da luz na realidade, embora  é englobado há registros e se  Vejamos, posso falar um pouco sobre  Como você vai acompanhar esses vídeos hoje estou  gravando o vídeo 5 que é o  operação leve na próxima  Eu só vou explicar o que é isso  são os registradores e os buffers após  vamos ver o micro código a linguagem  máquina alguns exemplos sobre  programação desta máquina e bem e  colocar até lá ficaria mais ou menos  explicação de como funciona  este computador

%----------------------------------------------------------------------------------------
%	CHAPTER SIX
%----------------------------------------------------------------------------------------

\chapter{Registros e Buffers}
bem-vindo ao capítulo 6 sobre o  operação de um computador 4  bits  O assunto de hoje são registros e  buffer sem mais introdução vamos ver  Do que se trata  Vamos começar com os bastardos, que é mais  simples este computador tem  apenas 2 buffers que são este circuito  integrado aquele da came que é aquele  Permite conectar a memória do programa  para o barramento de dados e este outro circuito  estadia integrada aqui que é aquela  permite que o calc alimente a saída do  acumulador na luz é na verdade o  mesmo circuito integrado porque se virmos  na folha de dados  isso vamos ver o nome no circuito  dentro é de 74 horas 244 a descrição  diz que é um octal basf airlines  motorista de 3 estados, quase vemos o  camada civil de saída e informa o motorista do ônibus  precisamente qual é o uso que vamos dar  nós  Como eu disse a você, como você pode ver isso?  é que metade são 2 vai ser 4 bits  então com um único integrado já temos  descobri o que precisamos para isso  computador como vimos em outros vídeos  em outros capítulos veremos  um circuito simplificado tal que  entenda melhor  muito bem vamos jogar  simulação colocamos tudo isso em zero  para colocar algum valor e vemos que  temos quatro valores de entrada e  quatro valores de saída enquanto o  ativar chip  desculpe-me ativar o pino está ativado  valor alto na saída não tenho nada  não tenho nem 0 e 1 digamos que não tenho  sem nível alto nível baixo  quando passo o chip de habilitação  no valor baixo ele me mostra na saída o que  que coloquei na entrada  a operação de um buffer é tão  simples assim mas para mais simples  seja o que for é realmente muito importante e  Agora vou explicar porque  o porquê bem fácil de entender  os computadores têm barramentos diferentes  Nesse caso, esse azul que vemos aqui é  um barramento de dados e qual é o problema se  Eu por exemplo este buffer a função  que cumpre é fazer esses quatro  bits mais baixos de memória  o programa passa para o barramento de dados, mas  quando eles passam quando dentro deste pino  de controle que tenho em um nível baixo  por que isso é tão importante porque  existem outros elementos do computador  conectou este barramento de dados conforme  exemplo, podemos ver a porta de entrada  e eu não posso ler simultaneamente o que  eu tenho na porta de entrada e li  o que me dá memória de dados porque  porque se por exemplo eu tiver um 0 aqui  e no outro pedaço da memória do  tamanho manu não é a definição de um  curto-circuito digamos assim o  as coisas têm que ser ativadas uma a uma  às vezes eles não podem estar trabalhando  simultaneamente essa é a operação  de hackers  este computador tem dois tipos de  registros temos todos estes integrados  estes que vêem aqui estes cinco que são 74  hc 573 e o acumulador que é um 74hs  574  vamos ver as fichas técnicas destes  integrado 74 573 diz que é um total  de volta aos estados livres de borda  enquanto o 574 diz que é um  caldo de chinelos positivos sestriere  série de estado bem, talvez isso não vai  fala muito mas é muito fácil de entender  quando eles veem o circuito simplificado e  sua operação  vamos ver  muito bem vamos começar com 573 como vemos  aqui está ok como eu disse isso quer dizer  que tem 8 bits nós só  usamos quatro desses bits porque  vamos lembrar que estamos olhando para um  computador de 4 bits e veja aqui é  temos na entrada todos os zeros e em  a saída nada é igual ao buffer do  habilitar chip habilitar chip  funciona quando coloco no nível baixo  Digamos que no zero eu coloco isso no nível  Eu desço e habilito a saída  e vimos que ele disse que tinha vamos de  de volta aos dados, diz transparente  o dge o que isso significa  é o que eu vou explicar  vou mudar os valores do  entrada e buffer diferente  olha aqui a saída  não tenho não troco valor  porque é isso porque para ele mudar o  bola na largada tem que estar dentro  alto lazio é esse alfinete que diz l  Eu olhei para os degraus altos e só então eu  passar os valores da entrada para o  saída se enquanto eu tiver o pino é  em alta, mude os valores de entrada  como estamos vendo também mudando  as saídas, mas se eu passar um valor  sob eu coloco em zero pode variar  quantas vezes é o valor da entrada  que a saída não descubra e do mesmo  para que o buffer repita quando eu  Eu coloquei isso em um desses valores não  não os coloca nem altos nem baixos  desative-o diretamente  ou está indo muito rápido  este é o último integrado que tivemos  falado é 574  vamos voltar para a folha de dados  A tosse gerada diz que é um flip vital  flops esse é o tipo de flow que você usa  a reta positiva de três estados  também veremos que diferença faz  entre o circuito anterior e este  como vemos da mesma forma enquanto  esse valor está em 1 na saída eu não  não tenho nada coloco no zero e me mostra  na saída o que eu tenho aqui agora eu vou  para alterar esses valores  da mesma forma que o integrado  anterior vemos que altero os valores  da entrada e na saída não me coloca  algum  mas o que acontece agora eu ativo esse valor  e se eu mudar esses valores digamos que  por enquanto é exatamente o mesmo que o  integrado acima, mas preste atenção  ao que vou mostrar agora vou  altere os valores de entrada e aqui  nada me muda enquanto o outro  integrado enquanto tinha lazio  ativado se menos alterado de mais de  isso não é um lazio é um clube que  Como diz na descrição diz que  isso significa que ele muda o valor com  a borda de subida que é quando  vai de 0 a positivo com um pulso de  assista olha agora vou passar pra  0 também não vai acontecer nada vem ver  permanece o mesmo, mas quando vai de 0 a 1  o pulso do relógio realmente está aqui comigo  mudar a aparência dos dados eu clico e  observe que ele muda imediatamente o  dados e quando volto ao zero não acontece nada  digamos que esta é a pequena diferença  que na verdade não é tão pequeno porque  muda completamente o funcionamento e  após o mesmo como tal como o  amortecedor  e que este outro 573 com o alfinete de  habilitação quando eu tenho em um não  me mostra nada é dizer que esses  registros vamos voltar para o circuito completo  Eles também atuam como um buffer.  Agora vamos ver alguns detalhes de  este circuito levante estes para entender  muito melhor no próximo capítulo do que  Vou explicar o microcódigo, mas  Vamos ver por exemplo o rd que é não sei se  lembre-se daquele registro que é usado  para endereços, use apenas um  sinal de controle que no do dge é  diga que está sempre ativado, mas  com lazio só então eles mudam meu  dados porque é isso porque eu isso isso  estou usando não estou colocando nenhum  dados para este barramento, então não importa o que  está sempre habilitado a partir do chip  simplesmente o que eu faço é tirar  esses dados e mandá-los aqui para ter  o endereço é pra dizer que nunca se sabe  circuito encontrado seria  vamos ver quanto temos por exemplo  o porto de entrada  o in é exatamente o oposto  Eu tenho o pin sempre ativado  porque é uma entrada e o que eu faço  é habilitar e desabilitar o chip  porque você pode circular por  diga de alguma forma  o registro geral é a mistura dos  dois porque se você não lembra disso tem  para assistir o vídeo novamente  apresentação se você não sabe do que estou falando  falando o mesmo no próximo vídeo eu vou  explicar o micro código que eu acho  pode entender um pouco melhor  Você escolheu este general e ele serve tanto para  salvar um dado carregá-lo no busto  Como você obteve os dados que salvou?  as duas funções então eu preciso  dois pinos na borda e no pino de  qualificação  e nestes dois que são o porto de  Sair só estou interessado na Lazio  porque digamos que os dados nunca entram no  bus se eu precisar colocar dados em um barramento  sempre terei que usar o alfinete  autorização se eu tiver que extrair dados  Eu sempre vou ter que usar um ônibus  o H  e o 174 é só o acumulador  porque funciona relativamente semelhante  mas o que ele usou foi esse sinal de  relógio que havia explicado a ele que quando  estamos falando sobre o programa contra isso  pino que está aqui é exatamente o que  entrar no relógio acumulador  Bem, eu acho que não há mais nada para  explicar sobre woofers e registros 

%----------------------------------------------------------------------------------------
%	CHAPTER SEVEN
%----------------------------------------------------------------------------------------

\chapter{Microcódigo e Linguagem de Máquina}
bem-vindo ao sétimo tutorial sobre  operação de um computador 4  bits hoje vamos ver dois tópicos o micro  código e linguagem de máquina por nós criamos  dois temas em vez de apenas um, que é como  se acostumar com os vídeos é porque  essas duas questões estão tão intimamente  relataram que é impossível separá-los  Antes de começar a explicar  Eu gostaria que alguns fossem claros  definições para isso veremos  wikipedia que certamente irá  definir muito melhor do que eu linguagem  máquina a linguagem da máquina ou código  máquina é o sistema de código  interpreta diretamente blé pelo  microcircuito programável como o  microprocessador de um computador ou  microcontrolador de um stand-alone este  A linguagem é composta por um conjunto  de instruções que determinam ações  para ser levado pela máquina muito bem  está bem claro vamos ver agora  conjunto de instruções um conjunto de  repertório de instruções de  instruções conjunto de instruções ou  é um  é uma especificação que detalha o  instruções que uma unidade central de  processamento pode entender e executar  Eu acho que está bem claro agora  quando ele continua explicando me parece que ele vai  para ser muito mais claro  e, finalmente, gostaria de ver o que diz  Wikipédia sobre microcódigo  microcódigo um microcódigo ou micro  programa é o nome de uma série de  instruções ou estruturas de dados  envolvidos na implementação do  linguagem de máquina de alto nível  muitos processadores especialmente  micro programado o micro código é  armazenado na memória acesso ao armazenamento  muito rapidamente então abaixo diz o  design de microprocessador de propósito  geral conhece duas técnicas que levam  a uma classificação destes em dois  grupos primeiro grupo o  microprocessadores com fio aqueles que  tem uma unidade de controle  projetado especificamente no  silício para um conjunto de instruções  microchips de concreto mox  agendou aqueles que têm um  unidade de controle genérica ou pré-projetada  que implementam um jogo de  instruções ou outra, dependendo do  micro programa  bem exatamente no caso de mim  computador que eu projetei  pode-se dizer que embora seja  fisicamente é feito não use um  memória para mim é um microcódigo de  isso agora vamos ver porque  como a wikipedia lhe disse o microcódigo  é implementado em uma memória de muito  acesso rápido fisicamente no circuito  é isso que vemos por trás  e isso é apenas uma memória agora  vamos ver como essa memória funciona  para explicar como funciona, vamos  ir ver como sempre um circuito  simplificado  como vemos aqui temos a memória de  programa que já explicamos os 4 bits  mais baixos são os operandos que eles passam  diretamente pelo buffer já agora  Explicamos no vídeo antes do ônibus  de dados e estes são os 4 bits  abaixo qual é o opcode  é o que diz ao computador o que  o que você tem que fazer e esses 4 bits  entrar neste circuito integrado 74 hc  154 que é o que é chamado de  multiplexador ou decodificador aqui vemos o  folha de dados diz decodificador de 4 a 16 linhas  de multiplex para ser vamos passar circuito  simplificado para ver como isso funciona  integrado  havíamos dito que é de 4 a 16 o que  isso significa que ele tem 4 entradas que  são esses 4 bits depois que eles não têm  os 2 pinos de habilitação que são  conectado ao negativo e como funciona  Isso é muito simples  vemos que aqui estamos em 0 e quando  estamos em 0 isso apenas me coloca em  Negativo apenas 0 e o resto  as saídas as outras 14 saem do  mais 15 saídas com licença coloca-os para  positivo agora vamos ver eu vou colocar isso  em um e isso só me coloca em apenas  em menos 1 vamos colocar outro  valor eu sei por exemplo vamos  coloquei 7 e só me coloca no negativo  7 entende o conceito e se eu colocar  efe digamos que tudo ligado eles me colocaram  menos 15 é dizer que  para cada valor em decimal  para cada valor em binário desses 4  bits me fazem negativa a saída que  correspondem enquanto o resto  manter positivo  Agora vamos ver o circuito  maior  no circuito principal temos aqui o  de multiplexer que o que fazemos é  construir uma matriz nas horizontais  é cada uma das saídas de  multiplexador e cada uma dessas saídas  corresponde a um código de operações  como vemos aqui não em e em hasta  chegar ao jp 7 e bem e o último  Eu não tenho nada se vemos no arquivo  palavra é exatamente agora não  tem um jp z e o último nada  cada uma dessas linhas horizontais é  o próprio opcode  e então temos linhas verticais e  em quantas linhas verticais você nata  precisamos de tantas linhas verticais  como sinais de controle central  precisamos do que são todos esses sinais  o que vemos aqui  sinais de controle, por exemplo  Eu digo alguns aqui temos um homem se isso  Lembra quando eu expliquei que a luz era o que  que me fez escolher na luz se eu quisesse  fazer uma soma ou operação lógica  nano por exemplo  Eu não sei se eles estão me seguindo porque em  meios complexos de explicar isso como  tudo está interligado e pode ser  torná-lo meio difícil de explicar  vamos continuar  cada linha vertical tem um  puxe a resistência, ou seja, coloque-a  o valor padrão é positivo e como  tínhamos dito que este multiplexador  cada saída em negativo o que fazemos  é com esses diodos quando  interligamos uma linha com este diodo  e por exemplo esta saída é negativa I  mude o valor para o efeito que  temos aqui que é positivo para negativo  não sei se estão me seguindo e assim  É assim que todos os sinais são gerados.  ao controle  Não vou explicar cada um dos  operações porque seria tremendamente  vídeo chato e muito longo mas vamos lá  para revisar alguns, vamos começar com o  primeira nota  porque é aquele que não faz nada digamos sim  Você vai me explicar exatamente o que não é  não faz nada e sim porque é muito importante  defina que nada que valores eles tem  estar no negativo que valores eles têm  ser positivo para que  microprocessador nesse ciclo não  absolutamente nada vamos fazer um  revisão rápida, então não é muito  chato como vemos aqui temos um  sinal de controle  o que é rum ou tenho positivo  o clube do acumulador eu tenho que  negativo  Eu tenho a saída negativa agora  Eles estão me afetando como eu disse a eles  no capítulo anterior  todos estes integrados estão ativados  para negativo por esse motivo e aqui eu faço um  linha indicando que é negado então  o valor padrão destes que são  diga o valor padrão destes  sinais de controle tem que ser quando  é um pino de habilitação tem que ser  positivo e quando é um la dge que aquele  ser negativo, digamos que estamos fazendo  isso não faz nada isso deveria ser  negativo agora porque agora bem não  ele está fingindo que ficou aqui em um  operação é por isso que me mostra estes  valores e assim definimos  exatamente o que é nada dentro e dentro e  bem, é o mesmo porque isso sempre  faz adição e subtração, mas bom como  Também não vamos passar o clube para ele  nada acontece e como a saída da luz  ele também tem um buffer chamado  atual que vamos procurar o que antes  sinais de controle atuais  Ia ser positivo se a gente ver aqui  deus vai ser negativo então não  vá fazer  Não sei se ficou muito claro porque o  Eu realmente não pretendo explicar isso  absolutamente tudo porque a ideia é mais  que desce neste desce no circuito e  começar a tentar entendê-lo é  muito complicado é o de cada diodo onde  vai cada  para lidar com o sinal de controle, mas se  eles levam tempo, tenho certeza que eles vão  você pode entender que não é algo tão complexo  finalmente bom temos todos esses  sinais de controle que servem para fazer  as diferentes coisas do microprocessador  e então temos este circuito lógico  aqui embaixo é o fim desse circuito  lógico, olha, é tudo feito de  Dan se comporta como eu disse a eles que eles são  portões universais, então você pode  fazer qualquer lógica de  não se comporte nada  Eu uso isso para saltos, temos três  tipos de saltos o salto incondicional  dizer que os saltos estão indo a parte  do programa para outro quando o salto é  incondicional  se eu chamar essa instrução, ela pula  diretamente enquanto o  salto é condicional tem que ser cumprido  algumas dessas condições, por exemplo  se eu tiver o sinal de salto  condicional  carry e não tenho o carry ativado não  ele ativará o sinal de controle de carga  que é o que me permite pular do que  mesmo com salto condicional zero se  a instrução de salto é chamada  zero condicional e na saída do  aluno eu tenho um zero não vai pular  enquanto se eu colocar o salto  incondicional vai pular sempre que eu  Liguei para ele e esse sinal logan que tínhamos  falado outro dia esse é o destaque  O que faz o contador pular?  de programas  muito bom sobre o micro código não vou  explicar muito mais porque como eu disse a você  são um monte de resistores  diodos de falha do resistor e você tem que  leva algum tempo para entender, mas  se ele ainda estiver no circuito tenho certeza que  você pode entender com um pouco de tempo  e paciência não  Agora vamos continuar olhando para todos esses  instruções que estão aqui, vamos voltar para  ao documento de voz e vamos rever  rápido e quase todas as instruções que  instruções a definição  todos podem projetar seu processador  como quiser e seguir as  instruções e com base nisso faz o  micro código não faz nada é útil para  fazer atrasos ignorar e o operando que  significa que ele ignora o operando como  Eu disse que as instruções são regulares sim  use-os abaixo do conjunto de operandos  o que você coloca não importa isso significa  o que o cndh ignora  nan entra o operando o acumulador faz  adicione o operando ao acumulador de carga  para o operando no acumulador fora e  coloca o valor do acumulador no  porta de saída para o carro e coloque o  valor do acumulador na porta de saída  de in para carla o valor do porto de  insira a no acumulador  rd coloca o valor do acumulador no  registro de endereço desta coisa  expliquei na apresentação mas volto  para explicar por que provavelmente  eles se esqueceram de como eu lhes disse a memória  um programa neste computador pode  ter até 256 bytes  isso era  porque temos esses dois contadores  binário que são 8 bits no total é  dizer que só podemos acessar o  memória para 256 bytes e também há este  de interruptor que me permite selecionar  entre 32 programas  mas como é um computador 4  me morde quando tenho que dar um salto  com 4 bits não é suficiente para mim dizer  de que posição desses 256 bits eu quero  bytes eu quero ir  Então, como faço isso?  Eu uso um registro de rede onde o  posição de salto, por exemplo, vamos terminar  no salto incondicional na posição  r do operando e aí eu já tenho  os 8 bits que preciso para definir um  que lugar de memória eu quero ir  agora, quando vamos a um exemplo para  entender melhor a área coloca o valor  do acumulador no registrador  ld efe a coloca o valor registrado em  a área do acumulador é um registro de  uso geral, ou seja, funciona tanto  como uma entrada, a saída é um  tipo de memória e então temos o  três saltos o salto incondicional que  sempre pule o salto se você estiver  Ativou o carry e o salto condicional  e como vemos, a posição sempre pula  de memória rd mais o valor de  operativo  e eu sei que tudo que eu estou tentando  explicar pode parecer confuso e  Para deixar mais claro vou fazer um  pequeno programa para fazer este pequeno  programa vou usar essa tabela do excel  que é algum tipo de compilador vashem  blair quem fez isso não fui eu que fiz isso  este trabalho este trabalho foi feito por um  cavalheiro mexicano chamado Antonio Esquivel  que viu a apresentação do  computador gostou parece que o projeto  e fiz um compilador fiz vários  versões tivemos uma ida e volta e o  você realmente fez um excelente trabalho  eu realmente muito obrigado  acabei no facebook com ela linda  família  Ele é um homem de saúde de Potosí a  cidade que está localizada mais ou menos em  o centro do México para que possam ser localizados  Desde já muito obrigado Antonio por  teu trabalho  Dito isso, vamos continuar com o compilador.  o programa que eu vou fazer vai ser  vou fazer algo bem simples  contador que conta de 0 a efe no  porto de partida  então vamos ver como fazer  Vamos começar isso, olhe aqui no  compilador nós escolhemos a primeira operação  Vou carregar no acumulador um valor que  que vale a pena  no acumulador 3 agora vamos ver porque  que carga 3 no acumulador  Agora vou chamar a função deste rd  coloca o valor do acumulador no  registro de endereço  nós escolhemos então rd  como vemos o rei não era o operando por  portanto no operando não é necessário  não coloque nada  Agora vou carregar outro valor no  acumulador  a ideia e o valor que vamos carregar  acumulador agora é zero onde  vamos começar a contar  Seguido isso vou pegar esse valor que  Eu tenho no acumulador para a porta e vemos  Fora  coloca o valor do acumulador no  porto de partida  agora queremos fazer soma  a operação de soma culpa e grave  a operação de adição é água quanto  vamos adicionar um vai adicionar um  em um  e finalmente vamos fazer um salto  incondicional  jp  0  o que significa este 3 o que vai acontecer  vai fazer é primeiro  quando eu zerar ele volta para  esta posição aqui porque se você se lembra  posição baixa é o que eu tenho  o registro durante a operação em  a parte superior do endereço é a do  jp e o operando, então aqui temos um  zero e aqui temos um 3 agora como  trabalha este compilador diretamente para  capon um nome agora vou mudar  vou colocar o nome como contador  exemplo  ponto eu dou gerar arquivo e onde  eu tenho isso é uma macro  onde tenho a tabela do excel que gerou  este este arquivo contador  se eu for para um editor hexadecimal e o que  Eu abro o arquivo para abrir que eu tenho  com os projetos tutoriais  Microcódigo de computadores de 4 bits  perfeito aqui vemos como o  arquivo que fez como os programas ou  coloque direto esses valores aqui no  um contador hexadecimal  e bem, eu fiz o programa, mas com um  compilador como o que vemos aqui é  muito mais simples porque é mais gráfico  eu tenho todos esses nomes  valores e é mais fácil lê-lo  linguagem de máquina, mas como temos  Falamos 33 70 30 40 2100 perfeito  o mesmo que temos aqui 33 70 30 40  2100  finalmente estaremos carregando neste  computador na memória do programa  aquele contador que eu tenho  Tutoriais de computador de 4 bits  contraponto e agora vamos dar  jogar para que simule e vemos que aqui está  a conta vamos fazer isso vamos  mudar a velocidade do relógio também  para torná-lo um pouco mais lento  vamos colocá-lo em 5 hertz  1 2 3  4 e se vamos contar programas  Eu coloquei 0 0 1 2  vou desacelerar ainda mais  que você consegue ver porque se não sair  vamos fazer a função como temos 0 1 2 3  Quatro cinco  345 e eu sei que fazendo 345 345 isso  exatamente o que a conta faz  bem, eu acho que isso é tudo, não há muito  mais para explicar sobre o  operação deste computador, mas  não será o último vídeo porque meu  ideia é se você se lembra na apresentação  Eu fiz vários programas de exemplo  Eu gostaria de explicar como funciona em  esses programas de amostra para que permaneça  mais claro, melhor compreendido  função de computador, mas  que assim se faz funcionar

%----------------------------------------------------------------------------------------
%	CHAPTER EIGHT
%----------------------------------------------------------------------------------------

\chapter{(final): Exemplos de Programação}
bem-vindo ao oitavo e último capítulo  desta série de vídeos onde estou  explicando como um  computador de 4 bits que diz niell on  este último capítulo  vamos ver exemplos de programação  de que programa eles tratam são apenas  os shows que assistimos no  apresentação da série  vamos ver rapidamente o que eles eram  primeiro é um contador que vai de 0 a 0 a  efe efe  o segundo que vamos ver é um olá  mundo em um 16 lcd para o qual  Escrevo o mundo vou escrevê-lo  o terceiro programa é um controlador do  motor de passo que basicamente o que  controla sua direção  vemos que há virar de um lado e virar  para o outro  o quarto programa é a renda de  dados por um teclado de matriz numérica  vemos todo o 1 aparece o 1 no  porta a tocou o 8 aparece o 8 o  porta para o 0 e assim por diante  e o último programa da reprodução  para a melodia do meu querido tempo clemente  [Música]  muito bom entender esses programas  vamos usar o compilador que fez  em excel o amigo mexicano antonio  esquivar e vamos começar com isso  exemplo do contador que vai de 00  até efe efe e bom e volte  continuar assim indefinidamente  Vamos ver como isso funciona  aqui temos contador  no primeiro endereço de memória  fazemos a ideia 4  ld lembrar que o que ele faz é carregar  o operando no acumulador, ou seja,  que no acumulador colocamos o valor 4  agora vamos ver porque  segundo comando network r e o que ele faz  é colocar o valor do acumulador é  dizer os 4 que tivemos no registro  de direção  Agora eu te lembro que você se registrou  endereçar o endereço de registro quando  damos um salto de um dos  esses três saltos que estão aqui o  incondicional o salto se ele correu o  pule se você estiver na posição em que  salto é definido por rd que na parte  diminuir mais o operando vamos ver isso em  o editor hexadecimal que se parece com um  um pouco mais claro  É como o que temos aqui vai  de 0 a f esta seria a parte inferior  ou seja, é o que o rd diz onde  pular enquanto as linhas seriam o  parte alta o que o operando diz  Vamos continuar na terceira linha o que  que fazemos é carregar 0 no acumulador  então carregamos esse valor em r  com este comando r eu acho que o  valor cumulativo no Egito para o  lembro que registrou é um record  de uso geral, ou seja, é um  tipo de memória rne nós  podemos colocar 4 bits lá e também  então podemos pegá-los novamente  a quarta linha não é o que não faz  não vamos dizer nada depois veremos porque  Eu coloquei que não está lá  quinta linha do que isso significa  que ao valor do acumulador somamos 1  e o novo valor do acumulador é passado para  ser 0 1 1  e obtemos esse valor pela porta b  já que este é o valor baixo quando  começamos a contar começamos a contar  primeiro pela porta b  vamos continuar  então dizemos que damos um salto  condicional dizemos pular e carregar  para 1 o que significa se carrie por que  comece a contar quando efe + 1  o que sai no valor acumulado que  leva o acumulador é zero e é ativado  a perna de transporte digamos assim  transborda e é acionado a faixa atinge  ser 0 1 2 3 assim até efe e no  próximo passo a faixa é ativada e  onde pula exatamente onde pula é  para a posição 14 porque porque salta  condicionado ao valor alto que é um e  o outro valor é retirado do registro de  endereço que nós colocamos em 4  pule para 14 se não vemos no no no  editor hexadecimal está bem aqui  vamos voltar ao excel  e se bom e se não de uma forma que não  posição perdida incondicionalmente  zero é dizer que repete tudo de novo  é o mesmo  vamos continuar  na posição 14 o que ele faz é pegar  o valor que havíamos gravado no  registro de propósito geral este registro  e o que ele faz é adicionar um e eles saem  percebendo que me parece o valor que é  no registrador a é apenas o valor  do porto e o valor que esta levando  essa tela que está aqui  e se voltarmos ao balcão o que fazemos para  Adicionamos 1 a esse valor e o colocamos em  o registro  adicionamos 1 e salvamos novamente no  registre-se para comer porque vamos usá-lo e  então nós o tiramos pelo porto  depois de carregarmos no  acumulador zero porque porque temos  você tem que reiniciá-lo para recuperá-lo  porta b e damos um salto  incondicional para a posição zero é  diga que o que quer que repita isso  sai daqui exceto quando o carry chega  faz essa outra parte aqui e volta para  zero basicamente é todo o programa  é um programa muito simples  talvez eles estejam se perguntando e tudo mais  este espaço aqui está morto como era  vemos aqui no editor hexadecimal e no  a verdade é que sim porque porque isso  computador é tão simples que não tem  o que eles chamam de saltos para suas rotinas  Então eu prefiro deixar esses  espaços de memória mortos, embora não  fazer qualquer coisa para simplificar o assunto do  hardware e programação para o melhor  eles estão se perguntando mais do que  não explique que vamos passar para  veja isso não é daqui porque se colocar  este não está aqui porque quando eu voltar  executar um salto incondicional para zero  O que eu faço não é eu não pular para zero em  Na verdade, pulo aqui para quatro porque tenho  a posição baixa do registrador é aquela  dê aquele que salvei em terceiro e pule  aqui 4 a questão é que se eu não colocar isso  não vou perder um ciclo e sei lá  Mariah a soma é por isso que eu tenho que adicionar  esta nota  vamos passar rapidamente para o próximo  exemplo o que é olá mundo em um lcd  16 vejamos como o  lcd primeiro é lido pode ser usado  com 4 bits que estão conectados ao  porto de partida de e então você precisa  duas assinaturas na verdade uma que é aquela  está conectado à porta de saída 0  e o drs que está conectado na saída  porta 1  Não vou explicar em detalhes como  estas eleições funcionam porque não é  o que vem ao caso do vídeo, mas  vamos ver isso primeiro para usar o  lcd você tem que iniciá-lo seguindo este  sequência que a gente vê aqui na folha  dados são os lcds estes são os acessos  hash que tem o driver h de 44  70 e 80  uma folha 47 temos a iniciação que  basicamente você tem que enviar o  iniciação para usá-lo com 4 bits existe  o que mandar aqui esses 2  se diz  1 todos os outros para 0 e fazer um pulso  no  em que dizemos  então vamos ver muito rapidamente  porque eu não acho que é tão longo  vídeo o código que é bem longo  mas não vou explicar todo o código  explique mais ou menos do que se trata  porque este este programa a regra é  muito chato nada muito inteligente  basicamente o que fazemos é carregar  lembre-se que ele disse 11, ou seja, é 3  cobramos é avaliar o acumulador e o que  tiramos na porta b que é onde fica  conectou o barramento de dados e depois  o que fazemos é ligar e desligar  o pino mais baixo da porta para o qual eu sei que  está conectado à fibra em que o carregamos  acumulador 1 é retirado através da porta  veja como fazemos um carro e então  temos que redefini-lo porque  estamos enviando a ele um pulso que é o que  repetimos três vezes e repetimos tudo  O que você tem que fazer até chegar a isso?  que é a conclusão do  inicialização e configuração  lcd digamos que é algo muito chato  carvalho em acumulador tira-o para um  porto ou por outro enviando pulsos e  depois como é que este lcd funciona muito  Eu rapidamente explico para você  bem como trabalhar como 4 bits  temos que repassar temos aqui temos o  tabela com os personagens por exemplo  vamos ver o hd olá mundo o  maiúsculo tem se é codificado em 8  bits, mas usamos apenas 4 e como fazemos  se você definir esse valor aqui que é aqui que  seria um 4 na parte superior de  os 8 bits enquanto esse valor que  está aqui que seria um 8 no  parte inferior dos 4 bits  então o que é feito nós colocamos  a parte superior que dizia que era um 4 lo  carregamos o acumulador e depois  colocamos na porta b e então  nós geramos o pulso agora o pulso é  diferente porque tem que ser  liguei o rc e depois passamos o  parte superior, ou seja, é o acesso com a  h com o ou com todos os caracteres o  espaço e assim que terminarmos tudo  fazemos uma sequência semelhante para  excluí-lo e voltar ao início e último  nós temos essas três linhas que o que  que fazemos é carregar no acumulador 2  colocamos isso no registro de  direção leste 2 não faz sentido  porque ignora o operando e fazemos um  salto indica incondicional para o  posição 3 em que se diz  na parte superior que viria  ser 23  onde temos o h de 23 32 estaria em  olá mundo realidade diga o que  repita continuamente olá mundo  apague tudo e repita depois explique  o anterior mais ou menos pra você ver  com que frequência estamos trabalhando  trabalhamos algumas frequências muito baixas  embora isso esteja funcionando a 100 meg por  100 anos enquanto no del  contador eu acho que é ainda mais baixo  correndo para que você possa vê-lo  para que você possa ver o efeito  estamos correndo 20 anos  O próximo exemplo que veremos é  o motor de passo como vemos é  anexado à porta para um motor de passo  passo ou bipolar que está girando neste  caso no sentido horário estava instável a  pouco isso é para a simulação e para o seu  tempo porque estou gravando, mas eu deveria  ser um movimento fluido este exemplo é  um pouco mais interessante porque como  vemos que também estamos usando a porta  entrada agora que tudo está no lugar  o valor um, exceto o valor mais baixo que  Eu sei que varia entre 0 para a torção  sentido horário e 1 para rotação no sentido anti-horário  Vamos ver como este programa é feito.  muito simples a primeira instrução que  fazemos é colocar no acumulador o  valor 2 esse valor do acumulador  carregamos no cadastro de endereços  ou seja, a parte inferior de onde vamos  dar o salto  próximo passo o que fazemos é levar  o valor da porta de entrada e o e lo  colocar no acumulador  e o que fazemos agora é uma operação  lógica lan lembre-se que isso  computador só tem dois  operações uma lógica que é nand e uma  aritmética o que é av  Nós vamos mostrar a ele do que se tratava  operação lógica nanda esta é a mesa  basicamente o que ele faz é que  sempre que houver um zero coloque um 1  então a única maneira de aparecer um  0 é se ambos os valores forem 1  se você se lembra no circuito  tivemos  desculpe-me exemplo confuso, mas isso  temos todos os valores em um exceto  o valor do menor que pode variar  entre 0 e 1, dependendo do sentido de  torcer que queremos dar então sim  voltamos à mesa  o que o cativa é esse valor que  temos que pode ser 1 110 ou pode ser  senão 11 11 diga todas as quatro vezes 1  e ele faz apenas a operação  entre esse valor e esse  e diz a você que se esse valor for 0 17 tudo  é 1 a única maneira que o resultado do  acumulador ser 0 é ser  efe e efe  e o que diz se esse valor for f  é neste alto valor o bit que  indica o sentido do salto de rotação para o  posição 2 que é a posição  2 basicamente é igual a 22 porque  temos o dobro aqui mais os 2 que  havíamos cobrado no acumulador sim  vamos para a posição 22 o que me diz é  aquele zac que colocou no acomodado no  bola para kaká vemos em binário como  como fazer bem é como ficar animado  bobinas que é 1010 e enviar pelo  porta a tirá-lo da porta a e  então temos 34 nem todos seriam esses  se não escrevemos ou se colocamos, não está lá  aqui é exatamente o mesmo  porque temos 4 não que não façamos nada  porque estou interessado no movimento sendo  fluente então o que eu tenho que fazer  é deixar 4 nós e fazer o próximo passo  que é colocar as bobinas nesse valor  100 1 que é 9 em hexadecimal e é  repita a mesma coisa isso o que faz é isso faz  essas quatro etapas e está apenas dizendo que  Se não, quando toco, mudo o bit.  naquele instante não me faz o  Inversão de marcha, na verdade, lo  faz uma vez que ele completa todos os quatro  Passos  e quando eu completar todos os quatro  passos o que ele faz é um salto  incondicional a zero que vem a ser  zero na posição 0 2 realidade que é  aquele que leva o valor s e se vemos aqui  temos que quando eles fazem os saltos  nós temos o que ele faz é  pega o valor e tem uma série de  ciclos e para que o movimento seja  fluente eu tenho que colocar isso aqui também  série de ciclos por esta razão  nota aqui  então chegamos a isso  posição aqui que dissemos se é zero  o que dizer que a porta é tudo em um  acs insere esse ciclo enquanto se  não é zero faz um salto incondicional para  posição 4 qual posição 4 o  posição 42 porque tínhamos o 2 do  endereço do registrador mais os 4 do operando  e o que fazemos é o ciclo oposto  diga aqui estamos fazendo 956 e aqui  o que fazemos é 5 9 a 6 e depois  nós o fazemos pular de volta para o  posição 0, ou seja, para a posição 0 2  Se virmos no exemplo, eles verão que  precisamente se eu mudar isso a mudança  não faz isso imediatamente faz um  tempo de repetição de todo o ciclo  repita o ciclo e faça a mudança  basicamente todos os seguintes  exemplo é esse do teclado que é meu  preferido ao melhor liga-nos muito  atenção toque em um número e ele aparece  aqui não é grande coisa todo o ponto de  vista de programação parece-me o  mais interessante vamos ver a tabela  para entender porque eu te digo isso  aqui temos este é o programa  e o que estamos fazendo  carregamos no acumulador o valor 2 que  valor que colocamos no registro  direção, ou seja, a parte inferior  Onde vamos pular a seguir?  Carregamos no acumulador o valor 1 e  nós o enviamos para a porta b  vamos ver primeiro para entender um pouco  é assim que está conectado  muito importante pulei essa parte  na porta, veja, conectamos o que  são as linhas do teclado matricial o que  o que vamos fazer é uma varredura que vamos  enviar 1 2  48 ou seja estamos aprendendo as baterias  de um e depois para o porto de entrada  que conectamos são esses três  valores de coluna e o maior valor  alto, digamos que o valor de três é  conectado alto o que é feito é um  varrido e, em seguida, em pé, dependendo  que toquei no momento em que o  varredura é como o programa sabe o que  número de toque, como vemos, também existem alguns  pull down ou seja, os valores por  padrão são definidos e quando eu  toque seria colocado em alta  Dito isso, vamos continuar olhando o código.  muito bem então nós estávamos aqui o que  nós fazemos isso é o que eu  Eu chamo de sweep que é o 1 que colocamos  na porta b então vamos colocar o  2 então vamos colocar 4 é posso colocar  os 8  e então o que fazemos é pegar o  valor da porta de entrada e  colocamos no acumulador que é o  instrução  e porque eles disseram que é o exemplo que  Eu gosto mais porque este exemplo  faz muito uso da saúde  Como eu disse, este não tem apenas dois  operações uma lógica que hernán e um  aritmética qual é a soma sem lembrar  Eu disse a eles várias vezes em vários vídeos  que a operação nam é um portão  universal, isto é, com operações  nan você pode fazer qualquer tipo de  lógica  como saber aqui tem estou interessado em saber o que  apertei o botão então como faço  isso eu não faço nada o val eu faço o valor  que tenho no acumulador é dizer que  o que eu tirei do porto de entrada e que  valor eu faço na 9 então eu faço de  de volta às 9 e então eu faço nan  efe  e então eu dou um pulo se o  resultado salta para uma posição de  memória que é dada ao fundo  pelo que tenho no terceiro e aqui o  posição de memória 5 vamos ver um  pouco que é sobre isso vamos supor  que eu apertei o 1  isso quer dizer que eu tenho o 1 isso seria  pressionar o 1 significaria que é 100 1 ou  qual é o primeiro 1 porque eles são lembrados  no circuito temos o maior valor  sempre conectado ao positivo agora  Embora eu queira ver se esse valor que  está aqui em vermelho marcado também é 1  Então, vamos voltar para a mesa para  escreva o exemplo  Acho que pressionei esse valor o que  faço é dar esse valor por esse mesmo  valor pelo valor que estou vendo  a condição para ele se ele apertou não  qual é a resposta disso é sempre  que eu tenho um 0 com um 1 então é 0 1  1 0  novamente eu faço nan e esse valor é esse  é este 9 qual é o resultado disso é  11 11 e ao fazer e o valor que eu  deu 1 111 com 111 diga com efe o  resultado torna-se 0 se eu colocar  qualquer outro valor aqui por exemplo 100  10 e o resultado aqui nunca será  zero então dessa forma fazendo três  instruções não consigo entender a lógica  para saber qual tecla eu apertei como eu sei  que você fique conhecendo a coluna  determinar por isso e a linha do  determinado pela varredura que agora  estamos na primeira fila  então eu disse a eles que se é zero é  diga que eu apertei o 1 pule o  posição de memória 5 na parte superior e  a parte inferior é 252 e se formos aqui para 52  que não vamos encontrar que acabei de encontrar  descarregar no acumulador o valor 1 e  apresentou-o no porto e recorda em  a porta a é aquela que exibe o valor  que eu apertei no teclado  enquanto a porta b está com a capa  da realidade, é por isso que aqui você pode ver que  qualquer número aí o que eu sou  fazendo é a varredura 1248  mais ou menos acho que ficou entendido que podemos  saiba um pouco mais como isso funciona  o mesmo depois de colocar o que eu faço  mesmo para o zero para o retorno a  carregue o valor da porta de valor em  o cumulativo o valor que um  porta de entrada e faça outro teste  agora com esse valor que é para ver se  aperte a segunda coluna e depois  mesmo com esse valor que é aquele  Vem a ser ver se apertei o  terceira coluna  uma vez eu vi e apertei alguns daqueles  três colunas o que eu faço de novo  colocado na porta b o segundo valor  da varredura, que é 2, então eu faço o que  mesmo com a porta de colocado os 4 eu faço  todas as perguntas e, em seguida, com o  porta b colocada 8 e está me fazendo  os saltos  como se diz dispensas condicionais  para a bola de correspondência para isso  colocaríamos por exemplo 62 72 82  sempre termina em 2 porque é o valor  eu tenho na porta de endereço e  uma vez que termina a um salto  incondicional a 0, ou seja, volta a  a posição é que dá aquele  seria a posição 0 2 bem isso é tudo  no exemplo do teclado  Como último exemplo, veremos que em  melhor chamar um pouco mais de atenção  embora a verdade seja muito simples  ponto de vista de programação não é  o grande problema é o do áudio que  produz a melodia de forma incremental  já tinha avançado um pouco  apresentação esta não sei o som  estou gerando com o computador  realidade com o computador que eu sou  enviando as notas nos tempos de  vídeo e som gerados com um  sintetizador um sintetizador muito  simples, vamos ver como isso funciona  sintetizador queremos aqui está  conectado à porta vêm que dá quatro  acessos conectados a um multiplexador este  multiplexador é exatamente o mesmo que  vimos no microcódigo o mesmo  circuito integrado de interesse para o céu  o que ele faz depende do valor  Que eu aconteça com ele aqui, por exemplo, aconteceu em  valor 000 me coloca abaixo de zero  enquanto todo mundo os coloca  alto se eu passei para 111 isso me coloca em  menos de 15 anos, enquanto todos os outros  eles estão em alta  vire bem cada uma dessas pernas  pino 0 não está conectado a nada  é o silêncio enquanto a pata  outras pernas estão aqui bom sendo o  estou usando estão conectados um  transístor pnp a bs  327 coloquei um transistor pnp para usos  generais e que iria ativar isso é  operando na saturação e cortando quando  esse valor é baixo  e com estes dependendo de como vem  ativar esses transistores gerou um  as notas agora vamos ver como é mais  aqui coloco os valores por exemplo  este é o sol mais sério depois de termos  sol foi re mi fa e um sol forte  digamos um oitavo  vamos ver o som que ele gera 1 555  executando em modo estável, ou seja,  como um oscilador  e essa rede de resistências é a que  gerar as notas vamos ver um pouco  aqui fiz uma ex dele explicando isso  esta é a fórmula 555 funcionando  em modo estável  o que nos diz que a frequência é dada  por 144 dividiu a área mais 2 crb todos  que multiplicado por c  muito bem aqui temos o valor diz que  é um valor que eu defini que vem para  ser este capacitor vamos ver aqui  este capacitó agradeceu-lhe na mão por  deus então temos uma resistência de  8 k 2 do que a resistência que não se move  que é aquele que no circuito chama o  de cabeça para baixo e o que varia é real  então eles não se encaixam aqui basicamente  Eu defino as frequências de cada nota  Digamos que o famoso 440 e eu sabia disso  internet para ver cada nota o que  frequência corresponde à capacidade não  a rb mexe, não mexe  definido em 8 k 2 x de modo que estes  números não me dão negativo digamos que não  Eu posso colocar uma resistência negativa  se eu tivesse colocado outro valor para caber 5  Aqui vamos ver o que acontece se eu colocar 5k  nada acontece como se eu fosse colocar 10 cadáveres  vamos ver o que acontece  coloquei 10 kaká e esse valor me dá um  resistência negativa que não tem  senso  então definiu um resistor que  existe isso é  8 k 2 muito bom então como funciona  esta tabela basicamente estes valores  esta fórmula é um apuramento destes já  Que valor o real tem que assumir para  me dê a frequência que eu sou  colocando aqui então o que fazemos é  comece com este valor o maior valor  afiado  então o que fazemos com esse valor  esta resistência permanece fixa enquanto  Esta será a diferença entre o  valor anterior e o seguinte e assim sucessivamente  sucessivamente é menos é  é menos é  há menos isso e outros do que com  esses valores  só é possível sintonizar perfeitamente o  sintetizador digamos que este é o único  medidas fixas mil 968  no primeiro e depois dos valores  seguindo para que quando eu for  ativando estes as notas são ativadas novamente  um por um sempre gera o  frequência de saída  Agora vamos ver o programa do programa  muito simples como eu disse a você do ponto de vista  visualização de programação não tem nenhum  Não tem nada de muito interessante, vamos ver  como é feito basicamente  agora temos a partitura  e o que eu faço é definir  uma colcheia por exemplo é quatro  ciclos de máquina, digamos que um preto seria  oito um branco seria 16 e assim  sucessivamente por exemplo aqui está  primeira nota que tenho para enviar é um  2 que é aquele que dura o tempo de um  oitavo então o que eu faço em enviar  o 2 que é o que é seria um 4  conectado na porta na verdade eu tenho  colocar o valor 4 para que eu ligue  e então ele enviou um duplo silêncio no caso  não, as notas iriam ficar então  ele mandou os 2 ele sempre mandou um silencio  Termino de enviar a nota que quero  com um silêncio para que as notas  eles estão separados e não há um preso a  o outro então bem basicamente é  que manda um 2 como uma cortina depois  enviou outra como colcheia enviou um C como  preto depois mandou esse sol grave que  é 1 tão preto quanto denia há oito  ciclos de máquinas e assim por diante nada  mais digamos que ele não tem nenhum  Ele não tem nada, vamos dar um branco  Blanca vem para fabricar máquinas de 16 séculos  e com isso gerou todas as notas  seguindo a pontuação  nada mais basicamente é o que parecia  algo complicado é na verdade algo muito  simples e isso por não gerar o  freqüentemente com ele com o computador  funciona muito devagar isso funciona isso  aqui eles estão dando para 15 aqui só  Com esse vídeo eu acabo com tudo  o que tem a ver  operação deste simples  Computador de 4 bits que diz eu espero  que este entendeu se não é assim  deixe nos comentários tentei  explicá-lo em algum outro vídeo ou  responda ali mesmo no comentário like  sempre no canal terão um link  para baixar todos os arquivos da tabela  de excel que o antonio esquivel fez e  bom tudo que você precisa para o próximo  faça seu próprio programa veja como  funciona e assim por diante  terminando bem esta série em  melhor do que eles podem deixar no  comentários tópicos que você estaria interessado em ver  minha ideia é continuar com o que esta com esp  32 e alguns outros projetos que tenho  à vista dizem nos comentários o que  tópicos que você gostaria de ver e bem, eu vou  tente desenvolvê-los

%----------------------------------------------------------------------------------------

\end{document}